% Options for packages loaded elsewhere
\PassOptionsToPackage{unicode}{hyperref}
\PassOptionsToPackage{hyphens}{url}
%
\documentclass[
]{article}
\title{Tarea 1}
\author{Ricardo Gabriel Rodriguez Gonzalez}
\date{26/1/2022}

\usepackage{amsmath,amssymb}
\usepackage{lmodern}
\usepackage{iftex}
\ifPDFTeX
  \usepackage[T1]{fontenc}
  \usepackage[utf8]{inputenc}
  \usepackage{textcomp} % provide euro and other symbols
\else % if luatex or xetex
  \usepackage{unicode-math}
  \defaultfontfeatures{Scale=MatchLowercase}
  \defaultfontfeatures[\rmfamily]{Ligatures=TeX,Scale=1}
\fi
% Use upquote if available, for straight quotes in verbatim environments
\IfFileExists{upquote.sty}{\usepackage{upquote}}{}
\IfFileExists{microtype.sty}{% use microtype if available
  \usepackage[]{microtype}
  \UseMicrotypeSet[protrusion]{basicmath} % disable protrusion for tt fonts
}{}
\makeatletter
\@ifundefined{KOMAClassName}{% if non-KOMA class
  \IfFileExists{parskip.sty}{%
    \usepackage{parskip}
  }{% else
    \setlength{\parindent}{0pt}
    \setlength{\parskip}{6pt plus 2pt minus 1pt}}
}{% if KOMA class
  \KOMAoptions{parskip=half}}
\makeatother
\usepackage{xcolor}
\IfFileExists{xurl.sty}{\usepackage{xurl}}{} % add URL line breaks if available
\IfFileExists{bookmark.sty}{\usepackage{bookmark}}{\usepackage{hyperref}}
\hypersetup{
  pdftitle={Tarea 1},
  pdfauthor={Ricardo Gabriel Rodriguez Gonzalez},
  hidelinks,
  pdfcreator={LaTeX via pandoc}}
\urlstyle{same} % disable monospaced font for URLs
\usepackage[margin=1in]{geometry}
\usepackage{color}
\usepackage{fancyvrb}
\newcommand{\VerbBar}{|}
\newcommand{\VERB}{\Verb[commandchars=\\\{\}]}
\DefineVerbatimEnvironment{Highlighting}{Verbatim}{commandchars=\\\{\}}
% Add ',fontsize=\small' for more characters per line
\usepackage{framed}
\definecolor{shadecolor}{RGB}{248,248,248}
\newenvironment{Shaded}{\begin{snugshade}}{\end{snugshade}}
\newcommand{\AlertTok}[1]{\textcolor[rgb]{0.94,0.16,0.16}{#1}}
\newcommand{\AnnotationTok}[1]{\textcolor[rgb]{0.56,0.35,0.01}{\textbf{\textit{#1}}}}
\newcommand{\AttributeTok}[1]{\textcolor[rgb]{0.77,0.63,0.00}{#1}}
\newcommand{\BaseNTok}[1]{\textcolor[rgb]{0.00,0.00,0.81}{#1}}
\newcommand{\BuiltInTok}[1]{#1}
\newcommand{\CharTok}[1]{\textcolor[rgb]{0.31,0.60,0.02}{#1}}
\newcommand{\CommentTok}[1]{\textcolor[rgb]{0.56,0.35,0.01}{\textit{#1}}}
\newcommand{\CommentVarTok}[1]{\textcolor[rgb]{0.56,0.35,0.01}{\textbf{\textit{#1}}}}
\newcommand{\ConstantTok}[1]{\textcolor[rgb]{0.00,0.00,0.00}{#1}}
\newcommand{\ControlFlowTok}[1]{\textcolor[rgb]{0.13,0.29,0.53}{\textbf{#1}}}
\newcommand{\DataTypeTok}[1]{\textcolor[rgb]{0.13,0.29,0.53}{#1}}
\newcommand{\DecValTok}[1]{\textcolor[rgb]{0.00,0.00,0.81}{#1}}
\newcommand{\DocumentationTok}[1]{\textcolor[rgb]{0.56,0.35,0.01}{\textbf{\textit{#1}}}}
\newcommand{\ErrorTok}[1]{\textcolor[rgb]{0.64,0.00,0.00}{\textbf{#1}}}
\newcommand{\ExtensionTok}[1]{#1}
\newcommand{\FloatTok}[1]{\textcolor[rgb]{0.00,0.00,0.81}{#1}}
\newcommand{\FunctionTok}[1]{\textcolor[rgb]{0.00,0.00,0.00}{#1}}
\newcommand{\ImportTok}[1]{#1}
\newcommand{\InformationTok}[1]{\textcolor[rgb]{0.56,0.35,0.01}{\textbf{\textit{#1}}}}
\newcommand{\KeywordTok}[1]{\textcolor[rgb]{0.13,0.29,0.53}{\textbf{#1}}}
\newcommand{\NormalTok}[1]{#1}
\newcommand{\OperatorTok}[1]{\textcolor[rgb]{0.81,0.36,0.00}{\textbf{#1}}}
\newcommand{\OtherTok}[1]{\textcolor[rgb]{0.56,0.35,0.01}{#1}}
\newcommand{\PreprocessorTok}[1]{\textcolor[rgb]{0.56,0.35,0.01}{\textit{#1}}}
\newcommand{\RegionMarkerTok}[1]{#1}
\newcommand{\SpecialCharTok}[1]{\textcolor[rgb]{0.00,0.00,0.00}{#1}}
\newcommand{\SpecialStringTok}[1]{\textcolor[rgb]{0.31,0.60,0.02}{#1}}
\newcommand{\StringTok}[1]{\textcolor[rgb]{0.31,0.60,0.02}{#1}}
\newcommand{\VariableTok}[1]{\textcolor[rgb]{0.00,0.00,0.00}{#1}}
\newcommand{\VerbatimStringTok}[1]{\textcolor[rgb]{0.31,0.60,0.02}{#1}}
\newcommand{\WarningTok}[1]{\textcolor[rgb]{0.56,0.35,0.01}{\textbf{\textit{#1}}}}
\usepackage{graphicx}
\makeatletter
\def\maxwidth{\ifdim\Gin@nat@width>\linewidth\linewidth\else\Gin@nat@width\fi}
\def\maxheight{\ifdim\Gin@nat@height>\textheight\textheight\else\Gin@nat@height\fi}
\makeatother
% Scale images if necessary, so that they will not overflow the page
% margins by default, and it is still possible to overwrite the defaults
% using explicit options in \includegraphics[width, height, ...]{}
\setkeys{Gin}{width=\maxwidth,height=\maxheight,keepaspectratio}
% Set default figure placement to htbp
\makeatletter
\def\fps@figure{htbp}
\makeatother
\setlength{\emergencystretch}{3em} % prevent overfull lines
\providecommand{\tightlist}{%
  \setlength{\itemsep}{0pt}\setlength{\parskip}{0pt}}
\setcounter{secnumdepth}{-\maxdimen} % remove section numbering
\ifLuaTeX
  \usepackage{selnolig}  % disable illegal ligatures
\fi

\begin{document}
\maketitle

\#Ejercicio 1 Se tiene el caso de tirar un dado no cargado. Defina lo
siguiente:

\begin{enumerate}
\def\labelenumi{\alph{enumi})}
\tightlist
\item
  ¿Cual es la variable aleatoria que representa la cara del dado que
  sale hacia arriba?
\end{enumerate}

\begin{Shaded}
\begin{Highlighting}[]
\NormalTok{w }\OtherTok{\textless{}{-}} \FunctionTok{c}\NormalTok{(}\DecValTok{1}\SpecialCharTok{:}\DecValTok{6}\NormalTok{)}
\end{Highlighting}
\end{Shaded}

\begin{enumerate}
\def\labelenumi{\alph{enumi})}
\setcounter{enumi}{1}
\tightlist
\item
  ¿Cuál es la fdp?
\end{enumerate}

\begin{Shaded}
\begin{Highlighting}[]
\NormalTok{fdp }\OtherTok{\textless{}{-}} \DecValTok{1}\SpecialCharTok{/}\DecValTok{6}
\end{Highlighting}
\end{Shaded}

\begin{enumerate}
\def\labelenumi{\alph{enumi})}
\setcounter{enumi}{2}
\tightlist
\item
  ¿Y cual es la FDA?
\end{enumerate}

\begin{Shaded}
\begin{Highlighting}[]
\NormalTok{FDA }\OtherTok{\textless{}{-}} \ControlFlowTok{function}\NormalTok{(x)\{}
  \FunctionTok{sum}\NormalTok{(x}\SpecialCharTok{*}\NormalTok{fdp)}
\NormalTok{\}}
\end{Highlighting}
\end{Shaded}

\begin{enumerate}
\def\labelenumi{\alph{enumi})}
\setcounter{enumi}{3}
\tightlist
\item
  Calcule la media y la varianza.
\end{enumerate}

\begin{Shaded}
\begin{Highlighting}[]
\CommentTok{\#La media se calcula como sigue:}
\NormalTok{media }\OtherTok{\textless{}{-}} \FunctionTok{sum}\NormalTok{(w)}\SpecialCharTok{/}\FunctionTok{length}\NormalTok{(w)}
\FunctionTok{print}\NormalTok{(media)}
\end{Highlighting}
\end{Shaded}

\begin{verbatim}
## [1] 3.5
\end{verbatim}

\begin{Shaded}
\begin{Highlighting}[]
\NormalTok{varianza }\OtherTok{\textless{}{-}}\NormalTok{ ((w }\SpecialCharTok{{-}}\NormalTok{ media)}\SpecialCharTok{\^{}}\DecValTok{2}\NormalTok{)}\SpecialCharTok{/}\FunctionTok{length}\NormalTok{(w)}
\FunctionTok{print}\NormalTok{(varianza)}
\end{Highlighting}
\end{Shaded}

\begin{verbatim}
## [1] 1.04166667 0.37500000 0.04166667 0.04166667 0.37500000 1.04166667
\end{verbatim}

\#Ejercicio 2

Una aguja de longitudl tiene un pivote en el centro de un c ́ırculo,
cuyo di ́ametro es igual al.~Defina lo siguiente:

\begin{enumerate}
\def\labelenumi{\alph{enumi})}
\tightlist
\item
  ¿Cual es la variable aleatoria que representa la posicion de la aguja?
\end{enumerate}

x es la variable aleatoria continua que representa la posicion de la
aguja, 0 \textless= x \textless2pi

\begin{Shaded}
\begin{Highlighting}[]
\NormalTok{l }\OtherTok{\textless{}{-}} \DecValTok{10}
\NormalTok{x }\OtherTok{\textless{}{-}}\NormalTok{ l }\SpecialCharTok{*}\NormalTok{pi}
\end{Highlighting}
\end{Shaded}

\begin{enumerate}
\def\labelenumi{\alph{enumi})}
\setcounter{enumi}{1}
\tightlist
\item
  ¿Ćual es la fdp?
\end{enumerate}

\begin{Shaded}
\begin{Highlighting}[]
\NormalTok{fdp }\OtherTok{\textless{}{-}} \DecValTok{1}\SpecialCharTok{/}\NormalTok{(}\DecValTok{2}\SpecialCharTok{*}\NormalTok{pi)}
\end{Highlighting}
\end{Shaded}

\begin{enumerate}
\def\labelenumi{\alph{enumi})}
\setcounter{enumi}{2}
\tightlist
\item
  ¿Y cu ́al es la FDA?
\end{enumerate}

\begin{Shaded}
\begin{Highlighting}[]
\NormalTok{FDA }\OtherTok{\textless{}{-}}\NormalTok{ x}\SpecialCharTok{/}\NormalTok{(}\DecValTok{2}\SpecialCharTok{*}\NormalTok{pi)}
\end{Highlighting}
\end{Shaded}

\begin{enumerate}
\def\labelenumi{\alph{enumi})}
\setcounter{enumi}{3}
\tightlist
\item
  Calcule la media y la varianza.
\end{enumerate}

\begin{Shaded}
\begin{Highlighting}[]
\NormalTok{media }\OtherTok{\textless{}{-}}\NormalTok{ l}\SpecialCharTok{*}\NormalTok{pi}

\CommentTok{\#varianza \textless{}{-} (x {-} l*pi)\^{}2 * 1/(2*pi)dx}
\CommentTok{\#x\^{}2 {-}2*x*l*pi +l\^{}2*pi\^{}2}

\CommentTok{\#[x\^{}3/3 {-} x\^{}2*l*pi + l\^{}2*pi\^{}2*x][1/(2*pi)]}
\CommentTok{\#[(8*pi\^{}3)/3 {-} 4*pi\^{}2*l*pi +l\^{}2*pi\^{}2*2*pi][1/(2*pi)]}
\CommentTok{\#4*(pi\^{}2)/3 {-} 2*pi\^{}2*l + l\^{}2*pi\^{}2}
\end{Highlighting}
\end{Shaded}

\#Ejercicio 3

Las labores diarias de John Doe requieren hacer 10 viajes redondos por
autom ́ovil entre dos ciudades. Una vezque realiza los 10 viajes, el se
̃nor Doe puede descansar el resto del d ́ıa, una motivaci ́on
suficientemente buenapara exceder el l ́ımite de velocidad. La
experiencia muestra que hay 40\% de probabilidad de ser multado
porexceso de velocidad en cualquier viaje redondo.

\begin{enumerate}
\def\labelenumi{\alph{enumi})}
\item
  ¿Cu ́al es la probabilidad de que el d ́ıa termine sin una multa por
  exceso de velocidad?
\item
  Si cada multa por exceso de velocidad es de\$80, ¿cu ́al es la multa
  diaria promedio?
\end{enumerate}

\#Ejercicio 4 A un taller de reparaci ́on de motores peque ̃nos llegan
trabajos de reparaci ́on a raz ́on de 10 por d ́ıa.

\begin{enumerate}
\def\labelenumi{\alph{enumi})}
\item
  ¿Cu ́al es el n ́umero promedio de trabajos que se reciben a diario en
  el taller?
\item
  ¿Cu ́al es la probabilidad de que no lleguen trabajos durante
  cualquier hora, suponiendo que el taller est ́aabierto 8 horas al dia?
\end{enumerate}

\#Ejercicio 5

Los autom ́oviles llegan al azar a una gasolinera. El tiempo promedio
entre llegadas es de 2 minutos. Determinela probabilidad de que el
tiempo entre llegadas no exceda de 1 minuto.

\#Ejercicio 6 Los datos siguientes representan el periodo (en segundos)
necesarios para transmitir un mensaje. Construya unhistograma de
frecuencia adecuado para los datos.

\end{document}
