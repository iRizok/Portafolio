% Options for packages loaded elsewhere
\PassOptionsToPackage{unicode}{hyperref}
\PassOptionsToPackage{hyphens}{url}
%
\documentclass[
]{article}
\usepackage{amsmath,amssymb}
\usepackage{lmodern}
\usepackage{iftex}
\ifPDFTeX
  \usepackage[T1]{fontenc}
  \usepackage[utf8]{inputenc}
  \usepackage{textcomp} % provide euro and other symbols
\else % if luatex or xetex
  \usepackage{unicode-math}
  \defaultfontfeatures{Scale=MatchLowercase}
  \defaultfontfeatures[\rmfamily]{Ligatures=TeX,Scale=1}
\fi
% Use upquote if available, for straight quotes in verbatim environments
\IfFileExists{upquote.sty}{\usepackage{upquote}}{}
\IfFileExists{microtype.sty}{% use microtype if available
  \usepackage[]{microtype}
  \UseMicrotypeSet[protrusion]{basicmath} % disable protrusion for tt fonts
}{}
\makeatletter
\@ifundefined{KOMAClassName}{% if non-KOMA class
  \IfFileExists{parskip.sty}{%
    \usepackage{parskip}
  }{% else
    \setlength{\parindent}{0pt}
    \setlength{\parskip}{6pt plus 2pt minus 1pt}}
}{% if KOMA class
  \KOMAoptions{parskip=half}}
\makeatother
\usepackage{xcolor}
\IfFileExists{xurl.sty}{\usepackage{xurl}}{} % add URL line breaks if available
\IfFileExists{bookmark.sty}{\usepackage{bookmark}}{\usepackage{hyperref}}
\hypersetup{
  pdftitle={Tarea 3},
  pdfauthor={Ricardo Gabriel Rodriguez Gonzalez},
  hidelinks,
  pdfcreator={LaTeX via pandoc}}
\urlstyle{same} % disable monospaced font for URLs
\usepackage[margin=1in]{geometry}
\usepackage{color}
\usepackage{fancyvrb}
\newcommand{\VerbBar}{|}
\newcommand{\VERB}{\Verb[commandchars=\\\{\}]}
\DefineVerbatimEnvironment{Highlighting}{Verbatim}{commandchars=\\\{\}}
% Add ',fontsize=\small' for more characters per line
\usepackage{framed}
\definecolor{shadecolor}{RGB}{248,248,248}
\newenvironment{Shaded}{\begin{snugshade}}{\end{snugshade}}
\newcommand{\AlertTok}[1]{\textcolor[rgb]{0.94,0.16,0.16}{#1}}
\newcommand{\AnnotationTok}[1]{\textcolor[rgb]{0.56,0.35,0.01}{\textbf{\textit{#1}}}}
\newcommand{\AttributeTok}[1]{\textcolor[rgb]{0.77,0.63,0.00}{#1}}
\newcommand{\BaseNTok}[1]{\textcolor[rgb]{0.00,0.00,0.81}{#1}}
\newcommand{\BuiltInTok}[1]{#1}
\newcommand{\CharTok}[1]{\textcolor[rgb]{0.31,0.60,0.02}{#1}}
\newcommand{\CommentTok}[1]{\textcolor[rgb]{0.56,0.35,0.01}{\textit{#1}}}
\newcommand{\CommentVarTok}[1]{\textcolor[rgb]{0.56,0.35,0.01}{\textbf{\textit{#1}}}}
\newcommand{\ConstantTok}[1]{\textcolor[rgb]{0.00,0.00,0.00}{#1}}
\newcommand{\ControlFlowTok}[1]{\textcolor[rgb]{0.13,0.29,0.53}{\textbf{#1}}}
\newcommand{\DataTypeTok}[1]{\textcolor[rgb]{0.13,0.29,0.53}{#1}}
\newcommand{\DecValTok}[1]{\textcolor[rgb]{0.00,0.00,0.81}{#1}}
\newcommand{\DocumentationTok}[1]{\textcolor[rgb]{0.56,0.35,0.01}{\textbf{\textit{#1}}}}
\newcommand{\ErrorTok}[1]{\textcolor[rgb]{0.64,0.00,0.00}{\textbf{#1}}}
\newcommand{\ExtensionTok}[1]{#1}
\newcommand{\FloatTok}[1]{\textcolor[rgb]{0.00,0.00,0.81}{#1}}
\newcommand{\FunctionTok}[1]{\textcolor[rgb]{0.00,0.00,0.00}{#1}}
\newcommand{\ImportTok}[1]{#1}
\newcommand{\InformationTok}[1]{\textcolor[rgb]{0.56,0.35,0.01}{\textbf{\textit{#1}}}}
\newcommand{\KeywordTok}[1]{\textcolor[rgb]{0.13,0.29,0.53}{\textbf{#1}}}
\newcommand{\NormalTok}[1]{#1}
\newcommand{\OperatorTok}[1]{\textcolor[rgb]{0.81,0.36,0.00}{\textbf{#1}}}
\newcommand{\OtherTok}[1]{\textcolor[rgb]{0.56,0.35,0.01}{#1}}
\newcommand{\PreprocessorTok}[1]{\textcolor[rgb]{0.56,0.35,0.01}{\textit{#1}}}
\newcommand{\RegionMarkerTok}[1]{#1}
\newcommand{\SpecialCharTok}[1]{\textcolor[rgb]{0.00,0.00,0.00}{#1}}
\newcommand{\SpecialStringTok}[1]{\textcolor[rgb]{0.31,0.60,0.02}{#1}}
\newcommand{\StringTok}[1]{\textcolor[rgb]{0.31,0.60,0.02}{#1}}
\newcommand{\VariableTok}[1]{\textcolor[rgb]{0.00,0.00,0.00}{#1}}
\newcommand{\VerbatimStringTok}[1]{\textcolor[rgb]{0.31,0.60,0.02}{#1}}
\newcommand{\WarningTok}[1]{\textcolor[rgb]{0.56,0.35,0.01}{\textbf{\textit{#1}}}}
\usepackage{graphicx}
\makeatletter
\def\maxwidth{\ifdim\Gin@nat@width>\linewidth\linewidth\else\Gin@nat@width\fi}
\def\maxheight{\ifdim\Gin@nat@height>\textheight\textheight\else\Gin@nat@height\fi}
\makeatother
% Scale images if necessary, so that they will not overflow the page
% margins by default, and it is still possible to overwrite the defaults
% using explicit options in \includegraphics[width, height, ...]{}
\setkeys{Gin}{width=\maxwidth,height=\maxheight,keepaspectratio}
% Set default figure placement to htbp
\makeatletter
\def\fps@figure{htbp}
\makeatother
\setlength{\emergencystretch}{3em} % prevent overfull lines
\providecommand{\tightlist}{%
  \setlength{\itemsep}{0pt}\setlength{\parskip}{0pt}}
\setcounter{secnumdepth}{-\maxdimen} % remove section numbering
\ifLuaTeX
  \usepackage{selnolig}  % disable illegal ligatures
\fi

\title{Tarea 3}
\author{Ricardo Gabriel Rodriguez Gonzalez}
\date{2022-03-16}

\begin{document}
\maketitle

\#\#Ejercicio 1.

Un agente de ventas realiza su trabajo en tres localidades A, B y C.
Para evitar desplazamientos innecesarios está todo el día en la misma
ciudad y allí pernocta, desplazándose a otra localidad al día siguiente,
si no tiene suficiente trabajo. después de estar trabajando un día en C,
la probabilidad de tener que seguir trabajando en ella al día siguiente
es 0.4, la de tener que viajar a B es 0.4, y la de tener que ir a A es
0.2. Si el viajante duerme un día en B, con probabilidad de 20\% tendrá
que seguir trabajando en la misma localidad al día siguiente en el 60\%
de los casos viajará a C mientras que irá a A con probabilidad 0.2. Por
último si el agente de ventas trabaja todo un día en A, permanecerá en
esa misma localidad, al día siguiente, con una probabilidad 0.1, irá a B
con una probabilidad 0.3 y a C con una probabilidad de 0.6.

\(X(t) =\)la ubicación del vendedor en el día t

Los estados: \(M = {A,B,C}\) Correspondientes a las tres diferentes
ciudades.

Temporalidad: Por día

\(\left[\begin{array}0 0.1,0.3,0.6\\0,2,0.2,0.6\\0.2,0.4,0.4\end{array}\right]\)

\begin{Shaded}
\begin{Highlighting}[]
\FunctionTok{library}\NormalTok{(markovchain)}
\end{Highlighting}
\end{Shaded}

\begin{verbatim}
## Package:  markovchain
## Version:  0.8.6
## Date:     2021-05-17
## BugReport: https://github.com/spedygiorgio/markovchain/issues
\end{verbatim}

\begin{Shaded}
\begin{Highlighting}[]
\NormalTok{statesNames }\OtherTok{=} \FunctionTok{c}\NormalTok{(}\StringTok{"A"}\NormalTok{,}\StringTok{"B"}\NormalTok{,}\StringTok{"C"}\NormalTok{)}
\NormalTok{mc\_p1 }\OtherTok{\textless{}{-}} \FunctionTok{new}\NormalTok{(}\StringTok{"markovchain"}\NormalTok{,}\AttributeTok{transitionMatrix =} \FunctionTok{matrix}\NormalTok{(}\FunctionTok{c}\NormalTok{(}\FloatTok{0.1}\NormalTok{,}\FloatTok{0.3}\NormalTok{,}\FloatTok{0.6}\NormalTok{,}\FloatTok{0.2}\NormalTok{,}\FloatTok{0.2}\NormalTok{,}\FloatTok{0.6}\NormalTok{,}\FloatTok{0.2}\NormalTok{,}\FloatTok{0.4}\NormalTok{,}\FloatTok{0.4}\NormalTok{),}\AttributeTok{byrow =} \ConstantTok{TRUE}\NormalTok{,}\AttributeTok{nrow =} \DecValTok{3}\NormalTok{,}\AttributeTok{dimnames =} \FunctionTok{list}\NormalTok{(statesNames,statesNames)))}
\NormalTok{mc\_p1}
\end{Highlighting}
\end{Shaded}

\begin{verbatim}
## Unnamed Markov chain 
##  A  3 - dimensional discrete Markov Chain defined by the following states: 
##  A, B, C 
##  The transition matrix  (by rows)  is defined as follows: 
##     A   B   C
## A 0.1 0.3 0.6
## B 0.2 0.2 0.6
## C 0.2 0.4 0.4
\end{verbatim}

Digarama de transición de estados:

\begin{Shaded}
\begin{Highlighting}[]
\FunctionTok{plot}\NormalTok{(mc\_p1)}
\end{Highlighting}
\end{Shaded}

\includegraphics{Tarea-3_files/figure-latex/unnamed-chunk-2-1.pdf} a) Si
hoy el vendedor está en C, Cúal es la probabilidad de que tambíen tenga
que trabajar en C al cabo de cuatro días?

\begin{Shaded}
\begin{Highlighting}[]
\CommentTok{\#Calculamos la matriz de 4 pasos}
\NormalTok{mc\_p1}\SpecialCharTok{\^{}}\DecValTok{4}
\end{Highlighting}
\end{Shaded}

\begin{verbatim}
## Unnamed Markov chain^4 
##  A  3 - dimensional discrete Markov Chain defined by the following states: 
##  A, B, C 
##  The transition matrix  (by rows)  is defined as follows: 
##        A      B      C
## A 0.1819 0.3189 0.4992
## B 0.1818 0.3190 0.4992
## C 0.1818 0.3174 0.5008
\end{verbatim}

\textbf{\emph{La probabilidad de que tambíen tenga que trabajar en C al
cabo de cuatro días es 0.5008 o 50.08\% }}

\begin{enumerate}
\def\labelenumi{\alph{enumi})}
\setcounter{enumi}{1}
\tightlist
\item
  Cuales son los porcentajes de días en los que el agente de ventas está
  en cada una de las tres localidades?
\end{enumerate}

\begin{Shaded}
\begin{Highlighting}[]
\FunctionTok{print}\NormalTok{(mc\_p1}\SpecialCharTok{\^{}}\DecValTok{7}\NormalTok{)}
\end{Highlighting}
\end{Shaded}

\begin{verbatim}
##           A         B         C
## A 0.1818181 0.3181755 0.5000064
## B 0.1818182 0.3181754 0.5000064
## C 0.1818182 0.3181882 0.4999936
\end{verbatim}

\begin{Shaded}
\begin{Highlighting}[]
\FunctionTok{print}\NormalTok{(mc\_p1}\SpecialCharTok{\^{}}\DecValTok{30}\NormalTok{)}
\end{Highlighting}
\end{Shaded}

\begin{verbatim}
##           A         B   C
## A 0.1818182 0.3181818 0.5
## B 0.1818182 0.3181818 0.5
## C 0.1818182 0.3181818 0.5
\end{verbatim}

\begin{Shaded}
\begin{Highlighting}[]
\FunctionTok{print}\NormalTok{(mc\_p1}\SpecialCharTok{\^{}}\DecValTok{365}\NormalTok{)}
\end{Highlighting}
\end{Shaded}

\begin{verbatim}
##           A         B   C
## A 0.1818182 0.3181818 0.5
## B 0.1818182 0.3181818 0.5
## C 0.1818182 0.3181818 0.5
\end{verbatim}

\textbf{\emph{Los porcentajes de días en los que el agente de ventas
está en cada una de las tres localidades son 18.18\%, 31.81\% y 50\%,
respectivamente.}}

\#\#Ejercicio 2.

Mi variable aleatoria es: \(X(T) = {\)Inspección de una computadora en
la hora t\(}\)

Los estados: \(M = {"Trabajando","Descompuesta"}\) Dos estados.

Temporalidad por hora.

\(\left[\begin{array}0 0.95,0.05\\0,5,0.5\end{array}\right]\)

\begin{Shaded}
\begin{Highlighting}[]
\NormalTok{statesNames }\OtherTok{=} \FunctionTok{c}\NormalTok{(}\StringTok{"Trabajando"}\NormalTok{,}\StringTok{"Descompuesta"}\NormalTok{)}
\NormalTok{mc\_p2 }\OtherTok{\textless{}{-}} \FunctionTok{new}\NormalTok{(}\StringTok{"markovchain"}\NormalTok{,}\AttributeTok{transitionMatrix =} \FunctionTok{matrix}\NormalTok{(}\FunctionTok{c}\NormalTok{(}\FloatTok{0.95}\NormalTok{,}\FloatTok{0.05}\NormalTok{,}\FloatTok{0.5}\NormalTok{,}\FloatTok{0.5}\NormalTok{),}\AttributeTok{byrow =} \ConstantTok{TRUE}\NormalTok{,}\AttributeTok{nrow =} \DecValTok{2}\NormalTok{,}\AttributeTok{dimnames =} \FunctionTok{list}\NormalTok{(statesNames,statesNames)))}
\NormalTok{mc\_p2}
\end{Highlighting}
\end{Shaded}

\begin{verbatim}
## Unnamed Markov chain 
##  A  2 - dimensional discrete Markov Chain defined by the following states: 
##  Trabajando, Descompuesta 
##  The transition matrix  (by rows)  is defined as follows: 
##              Trabajando Descompuesta
## Trabajando         0.95         0.05
## Descompuesta       0.50         0.50
\end{verbatim}

b)Encuentre el tiempo

\begin{Shaded}
\begin{Highlighting}[]
\FunctionTok{print}\NormalTok{( }\FunctionTok{meanFirstPassageTime}\NormalTok{(mc\_p2))}
\end{Highlighting}
\end{Shaded}

\begin{verbatim}
##              Trabajando Descompuesta
## Trabajando            0           20
## Descompuesta          2            0
\end{verbatim}

\#\#Ejercicio 3.

Mi variable aleatoria es: \(X(T) =\)La posición de la partícula dentro
de un circulot.

Los estados: \$M = \{``0'',``1'',``2'',``3'',``4''\}\$5 estados
correspondientes a los puntos marcados en el círculo.

Temporalidad: Por paso
\(\left[\begin{array}00,0.5,0,0,0.5\\0.5,0,0.5,0,0\\0,0.5,0,0.5,0\\0,0,0.5,0,0.5\\0.5,0,0,0.5,0\end{array}\right]\)

a)Encuentre la matriz de transición(de un paso)

\begin{Shaded}
\begin{Highlighting}[]
\NormalTok{statesNames }\OtherTok{=} \FunctionTok{c}\NormalTok{(}\StringTok{"0"}\NormalTok{,}\StringTok{"1"}\NormalTok{,}\StringTok{"2"}\NormalTok{,}\StringTok{"3"}\NormalTok{,}\StringTok{"4"}\NormalTok{)}
\NormalTok{mc\_p3 }\OtherTok{\textless{}{-}} \FunctionTok{new}\NormalTok{(}\StringTok{"markovchain"}\NormalTok{,}\AttributeTok{transitionMatrix =} \FunctionTok{matrix}\NormalTok{(}\FunctionTok{c}\NormalTok{(}\DecValTok{0}\NormalTok{,}\FloatTok{0.5}\NormalTok{,}\DecValTok{0}\NormalTok{,}\DecValTok{0}\NormalTok{,}\FloatTok{0.5}\NormalTok{,}\FloatTok{0.5}\NormalTok{,}\DecValTok{0}\NormalTok{,}\FloatTok{0.5}\NormalTok{,}\DecValTok{0}\NormalTok{,}\DecValTok{0}\NormalTok{,}\DecValTok{0}\NormalTok{,}\FloatTok{0.5}\NormalTok{,}\DecValTok{0}\NormalTok{,}\FloatTok{0.5}\NormalTok{,}\DecValTok{0}\NormalTok{,}\DecValTok{0}\NormalTok{,}\DecValTok{0}\NormalTok{,}\FloatTok{0.5}\NormalTok{,}\DecValTok{0}\NormalTok{,}\FloatTok{0.5}\NormalTok{,}\FloatTok{0.5}\NormalTok{,}\DecValTok{0}\NormalTok{,}\DecValTok{0}\NormalTok{,}\FloatTok{0.5}\NormalTok{,}\DecValTok{0}\NormalTok{),}\AttributeTok{byrow =} \ConstantTok{TRUE}\NormalTok{,}\AttributeTok{nrow =} \DecValTok{5}\NormalTok{,}\AttributeTok{dimnames =} \FunctionTok{list}\NormalTok{(statesNames,statesNames)))}
\FunctionTok{plot}\NormalTok{(mc\_p3)}
\end{Highlighting}
\end{Shaded}

\includegraphics{Tarea-3_files/figure-latex/unnamed-chunk-7-1.pdf} b)

\begin{Shaded}
\begin{Highlighting}[]
\CommentTok{\#elevar matriz a diferentes potencias}
\end{Highlighting}
\end{Shaded}

\begin{enumerate}
\def\labelenumi{\alph{enumi})}
\setcounter{enumi}{2}
\tightlist
\item
\end{enumerate}

\begin{Shaded}
\begin{Highlighting}[]
\CommentTok{\#utilizar la funcion de steadyStates() y comparar con inciso anterior}
\end{Highlighting}
\end{Shaded}

\#\#Ejercicio 4.

Mi variable aleatoria es: \(X(T) =\)La ubicación en la cuadrícula en el
lanzamiento t

Los estados: \(M = {"A","B","C","D"}\) 4 estados correspondientes a las
casillas.

Temporalidad: por lanzamiento
\(\left[\begin{array}0 1/6,2/6,2/6,1/6\\1/6,1/6,2/6,2/6\\2/6,1/6,1/6,2/6\\2/6,2/6,1/6,1/6\end{array}\right]\)

\begin{Shaded}
\begin{Highlighting}[]
\NormalTok{statesNames }\OtherTok{=} \FunctionTok{c}\NormalTok{(}\StringTok{"A"}\NormalTok{,}\StringTok{"B"}\NormalTok{,}\StringTok{"C"}\NormalTok{,}\StringTok{"D"}\NormalTok{)}
\NormalTok{mc\_p4 }\OtherTok{\textless{}{-}} \FunctionTok{new}\NormalTok{(}\StringTok{"markovchain"}\NormalTok{,}\AttributeTok{transitionMatrix =} \FunctionTok{matrix}\NormalTok{(}\FunctionTok{c}\NormalTok{(}\DecValTok{1}\SpecialCharTok{/}\DecValTok{6}\NormalTok{, }\DecValTok{2}\SpecialCharTok{/}\DecValTok{6}\NormalTok{, }\DecValTok{2}\SpecialCharTok{/}\DecValTok{6}\NormalTok{, }\DecValTok{1}\SpecialCharTok{/}\DecValTok{6}\NormalTok{, }\DecValTok{1}\SpecialCharTok{/}\DecValTok{6}\NormalTok{, }\DecValTok{1}\SpecialCharTok{/}\DecValTok{6}\NormalTok{, }\DecValTok{2}\SpecialCharTok{/}\DecValTok{6}\NormalTok{, }\DecValTok{2}\SpecialCharTok{/}\DecValTok{6}\NormalTok{, }\DecValTok{2}\SpecialCharTok{/}\DecValTok{6}\NormalTok{, }\DecValTok{1}\SpecialCharTok{/}\DecValTok{6}\NormalTok{, }\DecValTok{1}\SpecialCharTok{/}\DecValTok{6}\NormalTok{, }\DecValTok{2}\SpecialCharTok{/}\DecValTok{6}\NormalTok{, }\DecValTok{2}\SpecialCharTok{/}\DecValTok{6}\NormalTok{, }\DecValTok{2}\SpecialCharTok{/}\DecValTok{6}\NormalTok{, }\DecValTok{1}\SpecialCharTok{/}\DecValTok{6}\NormalTok{, }\DecValTok{1}\SpecialCharTok{/}\DecValTok{6}\NormalTok{),}\AttributeTok{byrow =} \ConstantTok{TRUE}\NormalTok{,}\AttributeTok{nrow =} \DecValTok{4}\NormalTok{,}\AttributeTok{dimnames =} \FunctionTok{list}\NormalTok{(statesNames,statesNames)))}
\NormalTok{mc\_p4}
\end{Highlighting}
\end{Shaded}

\begin{verbatim}
## Unnamed Markov chain 
##  A  4 - dimensional discrete Markov Chain defined by the following states: 
##  A, B, C, D 
##  The transition matrix  (by rows)  is defined as follows: 
##           A         B         C         D
## A 0.1666667 0.3333333 0.3333333 0.1666667
## B 0.1666667 0.1666667 0.3333333 0.3333333
## C 0.3333333 0.1666667 0.1666667 0.3333333
## D 0.3333333 0.3333333 0.1666667 0.1666667
\end{verbatim}

\begin{enumerate}
\def\labelenumi{\alph{enumi})}
\tightlist
\item
  Exprese el problema como una cadena de markov.
\end{enumerate}

\textbf{\emph{INCLUIR PRUEBA DE PROPIEDAD MARKOVIANA}}

b)Determine la ganancia o pérdida despues de lanzar el dado 5 veces

\begin{Shaded}
\begin{Highlighting}[]
\CommentTok{\#Se calcula la matriz de 5 pasos}
\NormalTok{m5p4 }\OtherTok{\textless{}{-}}\NormalTok{ mc\_p4}\SpecialCharTok{\^{}}\DecValTok{5}
\FunctionTok{print}\NormalTok{(m5p4)}
\end{Highlighting}
\end{Shaded}

\begin{verbatim}
##           A         B         C         D
## A 0.2502572 0.2497428 0.2497428 0.2502572
## B 0.2502572 0.2502572 0.2497428 0.2497428
## C 0.2497428 0.2502572 0.2502572 0.2497428
## D 0.2497428 0.2497428 0.2502572 0.2502572
\end{verbatim}

\begin{Shaded}
\begin{Highlighting}[]
\CommentTok{\#calcular la ganancia/pérdida esperados}
\NormalTok{Ct }\OtherTok{\textless{}{-}} \FunctionTok{c}\NormalTok{(}\DecValTok{4}\NormalTok{,}\SpecialCharTok{{-}}\DecValTok{2}\NormalTok{,}\SpecialCharTok{{-}}\DecValTok{6}\NormalTok{,}\DecValTok{9}\NormalTok{)}
\FunctionTok{print}\NormalTok{(}\FunctionTok{sum}\NormalTok{(m5p4}\SpecialCharTok{*}\NormalTok{Ct))}
\end{Highlighting}
\end{Shaded}

\begin{verbatim}
## [1] 5
\end{verbatim}

\textbf{\emph{La ganancia esperada despues de lanzr el dado 5 veces es
de \$5}}

\#\#Ejercicio 5.

Mi variable aleatoria es: \(X(T) =\)El paso de cal 1 de un estudiante

Los estados:
\(M = {"Modulo 1","Modulo 2","Modulo 3","Modulo 4","Cal 2"}\) 5 Modulos
de cal 1 e inicio de cal 2.

Temporalidad: por mes(4 Semanas)
\(\left[\begin{array}0 0.8,0.2,0,0,0\\0,0.78,0.22,0,0\\0,0,0.75,0.25,0\\0,0,0,0.7,0.3\\0,0,0,0,1\end{array}\right]\)

\begin{enumerate}
\def\labelenumi{\alph{enumi})}
\tightlist
\item
  Exprese el problema como una cadena de Markov. \textbf{\emph{Propiedad
  markoviana para los 5 estados }}
\end{enumerate}

\begin{Shaded}
\begin{Highlighting}[]
\FunctionTok{library}\NormalTok{(markovchain)}
\NormalTok{statesNames }\OtherTok{=} \FunctionTok{c}\NormalTok{(}\StringTok{"Modulo 1"}\NormalTok{,}\StringTok{"Modulo 2"}\NormalTok{,}\StringTok{"Modulo 3"}\NormalTok{,}\StringTok{"Modulo 4"}\NormalTok{,}\StringTok{"Cal 2"}\NormalTok{)}
\NormalTok{mc\_p5 }\OtherTok{\textless{}{-}} \FunctionTok{new}\NormalTok{(}\StringTok{"markovchain"}\NormalTok{,}\AttributeTok{transitionMatrix =} \FunctionTok{matrix}\NormalTok{(}\FunctionTok{c}\NormalTok{(}\FloatTok{0.8}\NormalTok{,}\FloatTok{0.2}\NormalTok{,}\DecValTok{0}\NormalTok{,}\DecValTok{0}\NormalTok{,}\DecValTok{0}\NormalTok{,}\DecValTok{0}\NormalTok{,}\FloatTok{0.78}\NormalTok{,}\FloatTok{0.22}\NormalTok{,}\DecValTok{0}\NormalTok{,}\DecValTok{0}\NormalTok{,}\DecValTok{0}\NormalTok{,}\DecValTok{0}\NormalTok{,}\FloatTok{0.75}\NormalTok{,}\FloatTok{0.25}\NormalTok{,}\DecValTok{0}\NormalTok{,}\DecValTok{0}\NormalTok{,}\DecValTok{0}\NormalTok{,}\DecValTok{0}\NormalTok{,}\FloatTok{0.7}\NormalTok{,}\FloatTok{0.3}\NormalTok{,}\DecValTok{0}\NormalTok{,}\DecValTok{0}\NormalTok{,}\DecValTok{0}\NormalTok{,}\DecValTok{0}\NormalTok{,}\DecValTok{1}\NormalTok{),}\AttributeTok{byrow =} \ConstantTok{TRUE}\NormalTok{,}\AttributeTok{nrow =} \DecValTok{5}\NormalTok{,}\AttributeTok{dimnames =} \FunctionTok{list}\NormalTok{(statesNames,statesNames)))}
\NormalTok{mc\_p5}
\end{Highlighting}
\end{Shaded}

\begin{verbatim}
## Unnamed Markov chain 
##  A  5 - dimensional discrete Markov Chain defined by the following states: 
##  Modulo 1, Modulo 2, Modulo 3, Modulo 4, Cal 2 
##  The transition matrix  (by rows)  is defined as follows: 
##          Modulo 1 Modulo 2 Modulo 3 Modulo 4 Cal 2
## Modulo 1      0.8     0.20     0.00     0.00   0.0
## Modulo 2      0.0     0.78     0.22     0.00   0.0
## Modulo 3      0.0     0.00     0.75     0.25   0.0
## Modulo 4      0.0     0.00     0.00     0.70   0.3
## Cal 2         0.0     0.00     0.00     0.00   1.0
\end{verbatim}

\begin{Shaded}
\begin{Highlighting}[]
\FunctionTok{plot}\NormalTok{(mc\_p5)}
\end{Highlighting}
\end{Shaded}

\includegraphics{Tarea-3_files/figure-latex/unnamed-chunk-12-1.pdf}

b)En promedio, un estudiante que inicio el modulo 1 al principio el
semestre actual ¿será capaz de llevar el módulo 2 el siguiente semestre?
(El cal 1 es un prerrequisito para el cal 2)

\begin{Shaded}
\begin{Highlighting}[]
\CommentTok{\#Debemos obtener la matriz a 5 pasos}
\FunctionTok{print}\NormalTok{(mc\_p5}\SpecialCharTok{\^{}}\DecValTok{5}\NormalTok{)}
\end{Highlighting}
\end{Shaded}

\begin{verbatim}
##          Modulo 1  Modulo 2  Modulo 3  Modulo 4     Cal 2
## Modulo 1  0.32768 0.3896256 0.2062455 0.0631499 0.0132990
## Modulo 2  0.00000 0.2887174 0.3770268 0.2262319 0.1080239
## Modulo 3  0.00000 0.0000000 0.2373047 0.3461734 0.4165219
## Modulo 4  0.00000 0.0000000 0.0000000 0.1680700 0.8319300
## Cal 2     0.00000 0.0000000 0.0000000 0.0000000 1.0000000
\end{verbatim}

\begin{Shaded}
\begin{Highlighting}[]
\CommentTok{\# La probabilidad que nos interesa es modulo 1 {-}\textgreater{} modulo 2}
\end{Highlighting}
\end{Shaded}

\textbf{\emph{En promedio, un estudiante que que inició el modulo 1 al
principio del semestre actual será capaz de llevar el módulo 2 el
siguiente semestre con una probabilidad de 0.3896 }}

\begin{enumerate}
\def\labelenumi{\alph{enumi})}
\setcounter{enumi}{2}
\tightlist
\item
  Un estudiante que haya completado sólo un módulo el semestre anterior
  ¿será capaz de terminar el cal 1 al final del semestre actual?
\end{enumerate}

\begin{Shaded}
\begin{Highlighting}[]
\CommentTok{\# Como el terminar cal 1 quiere decir que llegara al estado cal 2. Y como cal 2 es un estado absorbente, calculamos la probabilidad de absorbencia.}

\FunctionTok{print}\NormalTok{(}\FunctionTok{absorptionProbabilities}\NormalTok{(mc\_p5))}
\end{Highlighting}
\end{Shaded}

\begin{verbatim}
##          Cal 2
## Modulo 1     1
## Modulo 2     1
## Modulo 3     1
## Modulo 4     1
\end{verbatim}

\begin{Shaded}
\begin{Highlighting}[]
\FunctionTok{print}\NormalTok{(}\FunctionTok{meanAbsorptionTime}\NormalTok{(mc\_p5))}
\end{Highlighting}
\end{Shaded}

\begin{verbatim}
##  Modulo 1  Modulo 2  Modulo 3  Modulo 4 
## 16.878788 11.878788  7.333333  3.333333
\end{verbatim}

\textbf{\emph{El 1.3\% de los estudiantes que hayan completado el modulo
1 al principio del semestre será capaz de terminar el cal 1 al final de
semestre actual.El 10.80\% de los estudiantes que hayan completado el
modulo 2 al principio del semestre será capaz de terminar el cal 1 al
final de semestre actual. El 41.6\% de los estudiantes que hayan
completado el modulo 3 al principio del semestre será capaz de terminar
el cal 1 al final de semestre actual. El 83.1\% de los estudiantes que
hayan completado el modulo 4 al principio del semestre será capaz de
terminar el cal 1 al final de semestre actual.}}

\begin{enumerate}
\def\labelenumi{\alph{enumi})}
\setcounter{enumi}{3}
\tightlist
\item
  ¿Recoienda aplicar la idea del módulo a otras materias básicas?
  Explique.
\end{enumerate}

\begin{Shaded}
\begin{Highlighting}[]
 \FunctionTok{print}\NormalTok{(}\FunctionTok{steadyStates}\NormalTok{(mc\_p5))}
\end{Highlighting}
\end{Shaded}

\begin{verbatim}
##      Modulo 1 Modulo 2 Modulo 3 Modulo 4 Cal 2
## [1,]        0        0        0        0     1
\end{verbatim}

\begin{Shaded}
\begin{Highlighting}[]
\FunctionTok{print}\NormalTok{(mc\_p5}\SpecialCharTok{\^{}}\DecValTok{10}\NormalTok{)}
\end{Highlighting}
\end{Shaded}

\begin{verbatim}
##           Modulo 1   Modulo 2   Modulo 3   Modulo 4     Cal 2
## Modulo 1 0.1073742 0.24016424 0.26342485 0.19084901 0.1981877
## Modulo 2 0.0000000 0.08335776 0.19832445 0.23385656 0.4844612
## Modulo 3 0.0000000 0.00000000 0.05631351 0.14032995 0.8033565
## Modulo 4 0.0000000 0.00000000 0.00000000 0.02824752 0.9717525
## Cal 2    0.0000000 0.00000000 0.00000000 0.00000000 1.0000000
\end{verbatim}

\begin{Shaded}
\begin{Highlighting}[]
\FunctionTok{print}\NormalTok{(mc\_p5}\SpecialCharTok{\^{}}\DecValTok{15}\NormalTok{)}
\end{Highlighting}
\end{Shaded}

\begin{verbatim}
##            Modulo 1   Modulo 2   Modulo 3    Modulo 4     Cal 2
## Modulo 1 0.03518437 0.11117534 0.17520575 0.184380155 0.4940544
## Modulo 2 0.00000000 0.02406684 0.07849143 0.126817112 0.7706246
## Modulo 3 0.00000000 0.00000000 0.01336346 0.043079498 0.9435570
## Modulo 4 0.00000000 0.00000000 0.00000000 0.004747562 0.9952524
## Cal 2    0.00000000 0.00000000 0.00000000 0.000000000 1.0000000
\end{verbatim}

\begin{Shaded}
\begin{Highlighting}[]
\FunctionTok{print}\NormalTok{(mc\_p5}\SpecialCharTok{\^{}}\DecValTok{120}\NormalTok{)}
\end{Highlighting}
\end{Shaded}

\begin{verbatim}
##              Modulo 1     Modulo 2     Modulo 3     Modulo 4 Cal 2
## Modulo 1 2.348543e-12 2.235991e-11 9.511191e-11 2.326957e-10     1
## Modulo 2 0.000000e+00 1.125518e-13 8.179211e-13 2.542022e-12     1
## Modulo 3 0.000000e+00 0.000000e+00 1.017072e-15 5.084068e-15     1
## Modulo 4 0.000000e+00 0.000000e+00 0.000000e+00 2.580862e-19     1
## Cal 2    0.000000e+00 0.000000e+00 0.000000e+00 0.000000e+00     1
\end{verbatim}

\#Ejercicio 6. Un fabricante de videograbadoras está tan seguro de su
calidad que ofrece garantía de reposición total si un aparato falla
dentro de los dos primeros años. Con base en datos compilados, la
compañía ha notado que sólo 1\% de sus grabadoras fallan durante el
primer año, mientras que 5\% de ellas sobreviven el primer año pero
fallan durante el segundo. La garantía no cubre grabadoras ya
reemplazadas.

a)Formule la evolución del estado de una grabadora como una cadena de
Markov cuyos estados incluyen dos estados absorbentes que representan la
necesidad de cubrir la garantía o el hecho de que una grabadora
sobreviva el periodo de garantía. Después construya la matriz de
transición (de un paso).

b)Encuentre la probabilidad de que el fabricante tenga que cubrir una
garantía.

\#\#Ejercicio 7.

Mi variable aleatoria es: \(X(T) =\)La ubicación donde se enccuentra un
auto

Los estados: \(M = {"Phoenix","Denver","Chicago","Atlanta"}\) 4
estados/ciudades.

Temporalidad: Por semana
\(\left[\begin{array}0 0,0.2,0.6,0.2\\0,0,0.6,0.4\\0,0.5,0,0.5\\0.1,0.1,0.8,0\end{array}\right]\)

\begin{enumerate}
\def\labelenumi{\alph{enumi})}
\tightlist
\item
  Exprese la situación como una cadena de Markov.
\end{enumerate}

\begin{Shaded}
\begin{Highlighting}[]
\NormalTok{statesNames }\OtherTok{=} \FunctionTok{c}\NormalTok{(}\StringTok{"Phoenix"}\NormalTok{,}\StringTok{"Denver"}\NormalTok{,}\StringTok{"Chicago"}\NormalTok{,}\StringTok{"Atlanta"}\NormalTok{)}
\NormalTok{mc\_p7 }\OtherTok{\textless{}{-}} \FunctionTok{new}\NormalTok{(}\StringTok{"markovchain"}\NormalTok{,}\AttributeTok{transitionMatrix =} \FunctionTok{matrix}\NormalTok{(}\FunctionTok{c}\NormalTok{(}\DecValTok{0}\NormalTok{,}\FloatTok{0.2}\NormalTok{,}\FloatTok{0.6}\NormalTok{,}\FloatTok{0.2}\NormalTok{,}\DecValTok{0}\NormalTok{,}\DecValTok{0}\NormalTok{,}\FloatTok{0.6}\NormalTok{,}\FloatTok{0.4}\NormalTok{,}\DecValTok{0}\NormalTok{,}\FloatTok{0.5}\NormalTok{,}\DecValTok{0}\NormalTok{,}\FloatTok{0.5}\NormalTok{,}\FloatTok{0.1}\NormalTok{,}\FloatTok{0.1}\NormalTok{,}\FloatTok{0.8}\NormalTok{,}\DecValTok{0}\NormalTok{),}\AttributeTok{byrow =} \ConstantTok{TRUE}\NormalTok{,}\AttributeTok{nrow =} \DecValTok{4}\NormalTok{,}\AttributeTok{dimnames =} \FunctionTok{list}\NormalTok{(statesNames,statesNames)))}
\FunctionTok{print}\NormalTok{(mc\_p7)}
\end{Highlighting}
\end{Shaded}

\begin{verbatim}
##         Phoenix Denver Chicago Atlanta
## Phoenix     0.0    0.2     0.6     0.2
## Denver      0.0    0.0     0.6     0.4
## Chicago     0.0    0.5     0.0     0.5
## Atlanta     0.1    0.1     0.8     0.0
\end{verbatim}

\begin{Shaded}
\begin{Highlighting}[]
\FunctionTok{plot}\NormalTok{(mc\_p7)}
\end{Highlighting}
\end{Shaded}

\includegraphics{Tarea-3_files/figure-latex/unnamed-chunk-16-1.pdf}

\begin{Shaded}
\begin{Highlighting}[]
\FunctionTok{print}\NormalTok{(}\FunctionTok{summary}\NormalTok{(mc\_p7))}
\end{Highlighting}
\end{Shaded}

\begin{verbatim}
## Unnamed Markov chain  Markov chain that is composed by: 
## Closed classes: 
## Phoenix Denver Chicago Atlanta 
## Recurrent classes: 
## {Phoenix,Denver,Chicago,Atlanta}
## Transient classes: 
## NONE 
## The Markov chain is irreducible 
## The absorbing states are: NONE
## $closedClasses
## $closedClasses[[1]]
## [1] "Phoenix" "Denver"  "Chicago" "Atlanta"
## 
## 
## $recurrentClasses
## $recurrentClasses[[1]]
## [1] "Phoenix" "Denver"  "Chicago" "Atlanta"
## 
## 
## $transientClasses
## list()
\end{verbatim}

\begin{enumerate}
\def\labelenumi{\alph{enumi})}
\setcounter{enumi}{1}
\tightlist
\item
  Si la agencia inicia la semana con 100 autos en cada lugar, ¿cómo será
  la distribucion en dos semanas?
\end{enumerate}

\begin{Shaded}
\begin{Highlighting}[]
\FunctionTok{print}\NormalTok{(mc\_p7}\SpecialCharTok{\^{}}\DecValTok{2}\NormalTok{)}
\end{Highlighting}
\end{Shaded}

\begin{verbatim}
##         Phoenix Denver Chicago Atlanta
## Phoenix    0.02   0.32    0.28    0.38
## Denver     0.04   0.34    0.32    0.30
## Chicago    0.05   0.05    0.70    0.20
## Atlanta    0.00   0.42    0.12    0.46
\end{verbatim}

\begin{Shaded}
\begin{Highlighting}[]
\NormalTok{autos }\OtherTok{\textless{}{-}} \FunctionTok{c}\NormalTok{(}\DecValTok{100}\NormalTok{,}\DecValTok{100}\NormalTok{,}\DecValTok{100}\NormalTok{,}\DecValTok{100}\NormalTok{)}
\FunctionTok{print}\NormalTok{(autos}\SpecialCharTok{*}\NormalTok{(mc\_p7}\SpecialCharTok{\^{}}\DecValTok{2}\NormalTok{))}
\end{Highlighting}
\end{Shaded}

\begin{verbatim}
##      Phoenix Denver Chicago Atlanta
## [1,]      11    113     142     134
\end{verbatim}

\textbf{\emph{La distribución después de dos semanas será de 11, 113,
142y 134 autos para cada ciudad Phoenix, Denver, Chicago y Atlanta,
respectivamente}}

\begin{enumerate}
\def\labelenumi{\alph{enumi})}
\setcounter{enumi}{2}
\tightlist
\item
  Si cada lugar está diseñado para manejar un máximo de 110 autos,
  ¿habría a la larga un problema de disponibilidad de espacio en
  cualquiera de los lugares?
\end{enumerate}

\begin{Shaded}
\begin{Highlighting}[]
\NormalTok{probEstadoEstable }\OtherTok{\textless{}{-}} \FunctionTok{steadyStates}\NormalTok{(mc\_p7)}
\FunctionTok{print}\NormalTok{(autos}\SpecialCharTok{*}\NormalTok{probEstadoEstable)}
\end{Highlighting}
\end{Shaded}

\begin{verbatim}
##       Phoenix   Denver  Chicago  Atlanta
## [1,] 3.108348 24.42274 41.38544 31.08348
\end{verbatim}

\textbf{\emph{A la larga no habria un problema de disponibilidad de
espacio en cualquiera de los lugares.}}

\begin{enumerate}
\def\labelenumi{\alph{enumi})}
\setcounter{enumi}{3}
\tightlist
\item
  Determine el promedioi de semanas que transcurren antes de que un auto
  regrese a su lugar de origen.
\end{enumerate}

\begin{Shaded}
\begin{Highlighting}[]
\FunctionTok{print}\NormalTok{(}\FunctionTok{meanRecurrenceTime}\NormalTok{(mc\_p7))}
\end{Highlighting}
\end{Shaded}

\begin{verbatim}
##   Phoenix    Denver   Chicago   Atlanta 
## 32.171429  4.094545  2.416309  3.217143
\end{verbatim}

\textbf{\emph{El promedio de semanas que transcurren antes de que un
auto regrese a su lugar de origen es de 32, 4, 2 y 3 semanas,
respectivamente }}

\#\#Ejercicio 12.

Mi variable aleatoria es: \(X(T) =\)La ubicación donde se enccuentra un
auto

Los estados: \(M = {"Phoenix","Denver","Chicago","Atlanta"}\) 4
estados/ciudades.

Temporalidad: Por año
\(\left[\begin{array}0 0,0.2,0.6,0.2\\0,0,0.6,0.4\\0,0.5,0,0.5\\0.1,0.1,0.8,0\end{array}\right]\)

\begin{Shaded}
\begin{Highlighting}[]
\NormalTok{statesNames }\OtherTok{=} \FunctionTok{c}\NormalTok{(}\StringTok{"Buena"}\NormalTok{,}\StringTok{"Regular"}\NormalTok{,}\StringTok{"Mala"}\NormalTok{)}
\NormalTok{p\_p12 }\OtherTok{\textless{}{-}} \FunctionTok{new}\NormalTok{ (}\StringTok{"markovchain"}\NormalTok{, }\AttributeTok{transitionMatrix =}\NormalTok{ matrix }
\NormalTok{(}\FunctionTok{c}\NormalTok{(}\FloatTok{0.2}\NormalTok{,}\FloatTok{0.5}\NormalTok{,}\FloatTok{0.3}\NormalTok{,}\DecValTok{0}\NormalTok{,}\FloatTok{0.5}\NormalTok{,}\FloatTok{0.5}\NormalTok{,}\DecValTok{0}\NormalTok{,}\DecValTok{0}\NormalTok{,}\DecValTok{1}\NormalTok{), }\AttributeTok{byrow =} \ConstantTok{TRUE}\NormalTok{ , }\AttributeTok{nrow =} \DecValTok{3}\NormalTok{, }\AttributeTok{dimnames =} \FunctionTok{list}\NormalTok{(statesNames, statesNames)))}
\FunctionTok{plot}\NormalTok{(p\_p12)}
\end{Highlighting}
\end{Shaded}

\includegraphics{Tarea-3_files/figure-latex/unnamed-chunk-20-1.pdf}

\begin{Shaded}
\begin{Highlighting}[]
\FunctionTok{print}\NormalTok{(}\FunctionTok{summary}\NormalTok{(p\_p12))}
\end{Highlighting}
\end{Shaded}

\begin{verbatim}
## Unnamed Markov chain  Markov chain that is composed by: 
## Closed classes: 
## Mala 
## Recurrent classes: 
## {Mala}
## Transient classes: 
## {Buena},{Regular}
## The Markov chain is not irreducible 
## The absorbing states are: Mala
## $closedClasses
## $closedClasses[[1]]
## [1] "Mala"
## 
## 
## $recurrentClasses
## $recurrentClasses[[1]]
## [1] "Mala"
## 
## 
## $transientClasses
## $transientClasses[[1]]
## [1] "Buena"
## 
## $transientClasses[[2]]
## [1] "Regular"
\end{verbatim}

\begin{Shaded}
\begin{Highlighting}[]
\NormalTok{statesNames }\OtherTok{=} \FunctionTok{c}\NormalTok{(}\StringTok{"Buena"}\NormalTok{,}\StringTok{"Regular"}\NormalTok{,}\StringTok{"Mala"}\NormalTok{)}
\NormalTok{p1\_p12 }\OtherTok{\textless{}{-}} \FunctionTok{new}\NormalTok{ (}\StringTok{"markovchain"}\NormalTok{, }\AttributeTok{transitionMatrix =} \FunctionTok{matrix}\NormalTok{(}\FunctionTok{c}\NormalTok{(}\FloatTok{0.3}\NormalTok{,}\FloatTok{0.6}\NormalTok{,}\FloatTok{0.1}\NormalTok{,}\FloatTok{0.1}\NormalTok{,}\FloatTok{0.6}\NormalTok{,}\FloatTok{0.3}\NormalTok{,}\FloatTok{0.05}\NormalTok{,}\FloatTok{0.4}\NormalTok{,}\FloatTok{0.55}\NormalTok{), }\AttributeTok{byrow =} \ConstantTok{TRUE}\NormalTok{ , }\AttributeTok{nrow =} \DecValTok{3}\NormalTok{, }\AttributeTok{dimnames =} \FunctionTok{list}\NormalTok{(statesNames, statesNames)))}

\FunctionTok{plot}\NormalTok{(p1\_p12)}
\end{Highlighting}
\end{Shaded}

\includegraphics{Tarea-3_files/figure-latex/unnamed-chunk-20-2.pdf}

\begin{Shaded}
\begin{Highlighting}[]
\FunctionTok{print}\NormalTok{(}\FunctionTok{summary}\NormalTok{(p1\_p12))}
\end{Highlighting}
\end{Shaded}

\begin{verbatim}
## Unnamed Markov chain  Markov chain that is composed by: 
## Closed classes: 
## Buena Regular Mala 
## Recurrent classes: 
## {Buena,Regular,Mala}
## Transient classes: 
## NONE 
## The Markov chain is irreducible 
## The absorbing states are: NONE
## $closedClasses
## $closedClasses[[1]]
## [1] "Buena"   "Regular" "Mala"   
## 
## 
## $recurrentClasses
## $recurrentClasses[[1]]
## [1] "Buena"   "Regular" "Mala"   
## 
## 
## $transientClasses
## list()
\end{verbatim}

a)¿Cúales son los tiempos esperados de recurrencia de cada estado?

\begin{Shaded}
\begin{Highlighting}[]
\CommentTok{\#para la matriz P}
\FunctionTok{print}\NormalTok{(}\FunctionTok{meanRecurrenceTime}\NormalTok{(p\_p12))}
\end{Highlighting}
\end{Shaded}

\begin{verbatim}
## Mala 
##    1
\end{verbatim}

\begin{Shaded}
\begin{Highlighting}[]
\CommentTok{\#para la matriz P1}
\FunctionTok{print}\NormalTok{(}\FunctionTok{meanRecurrenceTime}\NormalTok{(p1\_p12))}
\end{Highlighting}
\end{Shaded}

\begin{verbatim}
##    Buena  Regular     Mala 
## 9.833333 1.903226 2.681818
\end{verbatim}

\begin{enumerate}
\def\labelenumi{\alph{enumi})}
\setcounter{enumi}{1}
\tightlist
\item
  Un jardín necesita dos sacos de fetilizante si la tierra es buena. La
  cantidad se incrementa en 25\% si la tierra es regular, y 60\% si la
  tierra es mala. El costo del fertilizante es de \$50 por saco. El
  jardinero estima un rendimiento anual de \$250 si no se utiliza
  fertilizante,
\end{enumerate}

\begin{Shaded}
\begin{Highlighting}[]
\NormalTok{costos}\OtherTok{\textless{}{-}} \DecValTok{50}\SpecialCharTok{*}\FunctionTok{c}\NormalTok{(}\DecValTok{1}\NormalTok{,}\FloatTok{1.25}\NormalTok{,}\FloatTok{1.6}\NormalTok{)}
\NormalTok{rendimientoAnualSF }\OtherTok{\textless{}{-}} \DecValTok{250} \CommentTok{\#sin fertilizante}
\NormalTok{rendimientoAnualCF }\OtherTok{\textless{}{-}} \DecValTok{420} \CommentTok{\#sin fertilizante}
\CommentTok{\#print(sum(costos*steadyStates(p1\_p12)))}

\FunctionTok{print}\NormalTok{(rendimientoAnualCF }\SpecialCharTok{{-}} \FunctionTok{sum}\NormalTok{(costos}\SpecialCharTok{*}\FunctionTok{steadyStates}\NormalTok{(p1\_p12)))}
\end{Highlighting}
\end{Shaded}

\begin{verbatim}
## [1] 352.2458
\end{verbatim}

\textbf{\emph{Es redituable utilizar fertilizante porque el rendimiento
anual seria de \$352 que es mayor a \$250 sin utilizar fertilizante.}}

c)Considere la matrix de transicion del jardinero con fertilizantes,
ccalcule el tiempo esperado de primera pasada desde los estados 2 y 3
(regular y mala) al estado 1 (bueno).

\begin{Shaded}
\begin{Highlighting}[]
\FunctionTok{print}\NormalTok{(}\FunctionTok{meanFirstPassageTime}\NormalTok{(p1\_p12))}
\end{Highlighting}
\end{Shaded}

\begin{verbatim}
##            Buena  Regular     Mala
## Buena    0.00000 1.774194 4.545455
## Regular 12.50000 0.000000 3.636364
## Mala    13.33333 2.419355 0.000000
\end{verbatim}

\textbf{\emph{El tiempo esperado desde el estado Regular a buena es de
12.5 años y de Mala a Regular es de 13.3 años.}}

\begin{enumerate}
\def\labelenumi{\alph{enumi})}
\setcounter{enumi}{3}
\tightlist
\item
  Considere la matriz de transicion del jardinero sin fertilizantes,
  calcule la probabilidad de absorcion al estado 3 (condición de tierra
  mala).
\end{enumerate}

\begin{Shaded}
\begin{Highlighting}[]
\FunctionTok{print}\NormalTok{(}\FunctionTok{absorptionProbabilities}\NormalTok{(p\_p12))}
\end{Highlighting}
\end{Shaded}

\begin{verbatim}
##         Mala
## Buena      1
## Regular    1
\end{verbatim}

\end{document}
