% Options for packages loaded elsewhere
\PassOptionsToPackage{unicode}{hyperref}
\PassOptionsToPackage{hyphens}{url}
%
\documentclass[
]{article}
\usepackage{amsmath,amssymb}
\usepackage{lmodern}
\usepackage{iftex}
\ifPDFTeX
  \usepackage[T1]{fontenc}
  \usepackage[utf8]{inputenc}
  \usepackage{textcomp} % provide euro and other symbols
\else % if luatex or xetex
  \usepackage{unicode-math}
  \defaultfontfeatures{Scale=MatchLowercase}
  \defaultfontfeatures[\rmfamily]{Ligatures=TeX,Scale=1}
\fi
% Use upquote if available, for straight quotes in verbatim environments
\IfFileExists{upquote.sty}{\usepackage{upquote}}{}
\IfFileExists{microtype.sty}{% use microtype if available
  \usepackage[]{microtype}
  \UseMicrotypeSet[protrusion]{basicmath} % disable protrusion for tt fonts
}{}
\makeatletter
\@ifundefined{KOMAClassName}{% if non-KOMA class
  \IfFileExists{parskip.sty}{%
    \usepackage{parskip}
  }{% else
    \setlength{\parindent}{0pt}
    \setlength{\parskip}{6pt plus 2pt minus 1pt}}
}{% if KOMA class
  \KOMAoptions{parskip=half}}
\makeatother
\usepackage{xcolor}
\IfFileExists{xurl.sty}{\usepackage{xurl}}{} % add URL line breaks if available
\IfFileExists{bookmark.sty}{\usepackage{bookmark}}{\usepackage{hyperref}}
\hypersetup{
  pdftitle={Tarea 4},
  pdfauthor={Ricardo Gabriel Rodriguez Gonzalez},
  hidelinks,
  pdfcreator={LaTeX via pandoc}}
\urlstyle{same} % disable monospaced font for URLs
\usepackage[margin=1in]{geometry}
\usepackage{color}
\usepackage{fancyvrb}
\newcommand{\VerbBar}{|}
\newcommand{\VERB}{\Verb[commandchars=\\\{\}]}
\DefineVerbatimEnvironment{Highlighting}{Verbatim}{commandchars=\\\{\}}
% Add ',fontsize=\small' for more characters per line
\usepackage{framed}
\definecolor{shadecolor}{RGB}{248,248,248}
\newenvironment{Shaded}{\begin{snugshade}}{\end{snugshade}}
\newcommand{\AlertTok}[1]{\textcolor[rgb]{0.94,0.16,0.16}{#1}}
\newcommand{\AnnotationTok}[1]{\textcolor[rgb]{0.56,0.35,0.01}{\textbf{\textit{#1}}}}
\newcommand{\AttributeTok}[1]{\textcolor[rgb]{0.77,0.63,0.00}{#1}}
\newcommand{\BaseNTok}[1]{\textcolor[rgb]{0.00,0.00,0.81}{#1}}
\newcommand{\BuiltInTok}[1]{#1}
\newcommand{\CharTok}[1]{\textcolor[rgb]{0.31,0.60,0.02}{#1}}
\newcommand{\CommentTok}[1]{\textcolor[rgb]{0.56,0.35,0.01}{\textit{#1}}}
\newcommand{\CommentVarTok}[1]{\textcolor[rgb]{0.56,0.35,0.01}{\textbf{\textit{#1}}}}
\newcommand{\ConstantTok}[1]{\textcolor[rgb]{0.00,0.00,0.00}{#1}}
\newcommand{\ControlFlowTok}[1]{\textcolor[rgb]{0.13,0.29,0.53}{\textbf{#1}}}
\newcommand{\DataTypeTok}[1]{\textcolor[rgb]{0.13,0.29,0.53}{#1}}
\newcommand{\DecValTok}[1]{\textcolor[rgb]{0.00,0.00,0.81}{#1}}
\newcommand{\DocumentationTok}[1]{\textcolor[rgb]{0.56,0.35,0.01}{\textbf{\textit{#1}}}}
\newcommand{\ErrorTok}[1]{\textcolor[rgb]{0.64,0.00,0.00}{\textbf{#1}}}
\newcommand{\ExtensionTok}[1]{#1}
\newcommand{\FloatTok}[1]{\textcolor[rgb]{0.00,0.00,0.81}{#1}}
\newcommand{\FunctionTok}[1]{\textcolor[rgb]{0.00,0.00,0.00}{#1}}
\newcommand{\ImportTok}[1]{#1}
\newcommand{\InformationTok}[1]{\textcolor[rgb]{0.56,0.35,0.01}{\textbf{\textit{#1}}}}
\newcommand{\KeywordTok}[1]{\textcolor[rgb]{0.13,0.29,0.53}{\textbf{#1}}}
\newcommand{\NormalTok}[1]{#1}
\newcommand{\OperatorTok}[1]{\textcolor[rgb]{0.81,0.36,0.00}{\textbf{#1}}}
\newcommand{\OtherTok}[1]{\textcolor[rgb]{0.56,0.35,0.01}{#1}}
\newcommand{\PreprocessorTok}[1]{\textcolor[rgb]{0.56,0.35,0.01}{\textit{#1}}}
\newcommand{\RegionMarkerTok}[1]{#1}
\newcommand{\SpecialCharTok}[1]{\textcolor[rgb]{0.00,0.00,0.00}{#1}}
\newcommand{\SpecialStringTok}[1]{\textcolor[rgb]{0.31,0.60,0.02}{#1}}
\newcommand{\StringTok}[1]{\textcolor[rgb]{0.31,0.60,0.02}{#1}}
\newcommand{\VariableTok}[1]{\textcolor[rgb]{0.00,0.00,0.00}{#1}}
\newcommand{\VerbatimStringTok}[1]{\textcolor[rgb]{0.31,0.60,0.02}{#1}}
\newcommand{\WarningTok}[1]{\textcolor[rgb]{0.56,0.35,0.01}{\textbf{\textit{#1}}}}
\usepackage{longtable,booktabs,array}
\usepackage{calc} % for calculating minipage widths
% Correct order of tables after \paragraph or \subparagraph
\usepackage{etoolbox}
\makeatletter
\patchcmd\longtable{\par}{\if@noskipsec\mbox{}\fi\par}{}{}
\makeatother
% Allow footnotes in longtable head/foot
\IfFileExists{footnotehyper.sty}{\usepackage{footnotehyper}}{\usepackage{footnote}}
\makesavenoteenv{longtable}
\usepackage{graphicx}
\makeatletter
\def\maxwidth{\ifdim\Gin@nat@width>\linewidth\linewidth\else\Gin@nat@width\fi}
\def\maxheight{\ifdim\Gin@nat@height>\textheight\textheight\else\Gin@nat@height\fi}
\makeatother
% Scale images if necessary, so that they will not overflow the page
% margins by default, and it is still possible to overwrite the defaults
% using explicit options in \includegraphics[width, height, ...]{}
\setkeys{Gin}{width=\maxwidth,height=\maxheight,keepaspectratio}
% Set default figure placement to htbp
\makeatletter
\def\fps@figure{htbp}
\makeatother
\setlength{\emergencystretch}{3em} % prevent overfull lines
\providecommand{\tightlist}{%
  \setlength{\itemsep}{0pt}\setlength{\parskip}{0pt}}
\setcounter{secnumdepth}{-\maxdimen} % remove section numbering
\ifLuaTeX
  \usepackage{selnolig}  % disable illegal ligatures
\fi

\title{Tarea 4}
\author{Ricardo Gabriel Rodriguez Gonzalez}
\date{2022-05-18}

\begin{document}
\maketitle

\hypertarget{ejercicio-1}{%
\subsection{Ejercicio 1}\label{ejercicio-1}}

La tienda de alimentos Mom-and-Pop's tiene un estacionamiento
peque\textasciitilde no con tres espacios reservados para los clientes.
Si la tienda esta abierta los autos llegan y usan un espacio con una
tasa media de 2 por hora. Para n = 0; 1; 2; 3, la probabilidad Pn de que
haya exactamente n espacios ocupados es P0 = 0:1, P1 = 0:2, P2 = 0:4, P3
= 0:3.

\begin{enumerate}
\def\labelenumi{\alph{enumi})}
\tightlist
\item
  Describa la interpretación de este estacionamiento como un sistema de
  colas. En particular, identifique los clientes y los servidores.
\end{enumerate}

\textbf{\emph{El sistema es el estacionamiento.Los clientes son los
autos. Los servidores son los espacios disponibles}}

¿Cuál es el servicio que se proporciona?

\textbf{\emph{Los clientes van a la tienda de alimientos Mom-and-Pop's a
comprar}}

¿Qué constituye el tiempo de servicio?

\textbf{\emph{El tiempo que pasa el cliente en la tienda (El tiempo que
están estacionados)}}

¿Cuál es la capacidad de la cola?

\textbf{\emph{La cola son los espacios reservados de estacionamiento
para los clientes de la tienda de alimientos Mom-and-Pop's. La capacidad
de la cola es de 3 autos}}

\begin{itemize}
\tightlist
\item
  Determine las medidas de desempeño básicas: L,Lq, W y Wq de este
  sistema de colas
\end{itemize}

Calculamos el número promedio de clientes en el sistema: \$L =
\displaystyle\sum\_\{n = 0\}\^{}3 n* P\_n = 0\emph{P\_0 + 1}P\_1 +
2\emph{P\_2 + 3}P\_3 = 0\emph{0.01 + 1}0.2+ 2\emph{0.4 + 3}0.3 = \$

\begin{Shaded}
\begin{Highlighting}[]
\NormalTok{n }\OtherTok{=} \FunctionTok{c}\NormalTok{(}\DecValTok{0}\NormalTok{,}\DecValTok{1}\NormalTok{,}\DecValTok{2}\NormalTok{,}\DecValTok{3}\NormalTok{)}
\NormalTok{Pn }\OtherTok{=} \FunctionTok{c}\NormalTok{(}\FloatTok{0.1}\NormalTok{,}\FloatTok{0.2}\NormalTok{,}\FloatTok{0.4}\NormalTok{,}\FloatTok{0.3}\NormalTok{)}
\NormalTok{L }\OtherTok{=} \FunctionTok{sum}\NormalTok{(n}\SpecialCharTok{*}\NormalTok{Pn)}
\FunctionTok{print}\NormalTok{(L)}
\end{Highlighting}
\end{Shaded}

\begin{verbatim}
## [1] 1.9
\end{verbatim}

Calculamos el tamaño (número de clientes) de la cola:

\begin{Shaded}
\begin{Highlighting}[]
\NormalTok{s }\OtherTok{=} \DecValTok{3}
\end{Highlighting}
\end{Shaded}

\$L\_q = \displaystyle\sum\_\{n = 3\}\^{}\infty (n-s)*P\_n \$

\begin{Shaded}
\begin{Highlighting}[]
\NormalTok{n }\OtherTok{=} \FunctionTok{c}\NormalTok{(}\DecValTok{3}\NormalTok{)}
\NormalTok{Lq }\OtherTok{=} \FunctionTok{sum}\NormalTok{((n}\SpecialCharTok{{-}}\NormalTok{s)}\SpecialCharTok{*}\NormalTok{Pn[}\DecValTok{3}\NormalTok{])}\CommentTok{\# 0 Clientes en la cola}
\FunctionTok{print}\NormalTok{(Lq)}
\end{Highlighting}
\end{Shaded}

\begin{verbatim}
## [1] 0
\end{verbatim}

*** Se esperan en promedio 0 autos en la cola. ***

Calculamos el tiempo que un cliente está en el sistema:

\begin{Shaded}
\begin{Highlighting}[]
\NormalTok{lambda }\OtherTok{=} \DecValTok{2} \CommentTok{\# 2 espacios/hora}
\end{Highlighting}
\end{Shaded}

\(W = L /\lambda = 1.9/2\)

\begin{Shaded}
\begin{Highlighting}[]
\NormalTok{W }\OtherTok{=}\NormalTok{ L}\SpecialCharTok{/}\NormalTok{lambda }\CommentTok{\#0.95 horas = 57 minutos}
\FunctionTok{print}\NormalTok{(W)}
\end{Highlighting}
\end{Shaded}

\begin{verbatim}
## [1] 0.95
\end{verbatim}

\begin{itemize}
\tightlist
\item
  El tiempo promedio que un auto pasa en el estacionamiento es de 57
  minutos. *
\end{itemize}

Calculamos el tiempo en la cola: \$W\_q = L\_q/\lambda \$

\begin{Shaded}
\begin{Highlighting}[]
\NormalTok{W\_q }\OtherTok{=}\NormalTok{ Lq}\SpecialCharTok{/}\NormalTok{lambda}
\FunctionTok{print}\NormalTok{(W\_q)}
\end{Highlighting}
\end{Shaded}

\begin{verbatim}
## [1] 0
\end{verbatim}

\begin{itemize}
\item
  El tiempo esperado que un auto pasa en la cola es de 0 horas. *
\item
  Use los resultados de b) para determinar el tiempo promedio que un
  auto permanece en el espacio. \(W - W_q = 57 - 0 = 57\)
\item
  El tiempo promedio que un auto permanece en el espacio de
  estacionamiento es de 57 minutos. *
\end{itemize}

\hypertarget{ejercicio-2}{%
\subsection{Ejercicio 2}\label{ejercicio-2}}

El Midtown Bank siempre tiene dos cajeras en servicio. Los clientes
llegan a las cajas a una tasa media de 40 por hora. Una cajera requiere
en promedio 2 minutos para servir a un cliente. Cuando ambas cajeras
están ocupadas, el cliente que llega se une a una cola y espera a que lo
atiendan. Por experiencia se sabe que los clientes esperan en la cola un
promedio de 1 minuto antes de pasar a la caja.

\begin{itemize}
\item
  Sistema: Banco, Clientes: clientes(personas) del banco, Servidores:
  cajeras *
\item
  Tasa de llegadas (\(\lambda\)) = 40 clientes/hora = 40/60
  clientes/minuto *
\item
  Tasa de servicio (\(\mu\))* = 2 minutos/cliente
\item
  Número de servidores = 2 *
\item
  Modelo: M/M/C *
\item
  Determine las medidas de desempeño básicas: \(W_q, W, L\) y \(L_q\) de
  este sistema.
\end{itemize}

\begin{Shaded}
\begin{Highlighting}[]
\NormalTok{lambda }\OtherTok{=} \DecValTok{40}\SpecialCharTok{/}\DecValTok{60}
\NormalTok{mu }\OtherTok{=} \DecValTok{2}
\NormalTok{s }\OtherTok{=} \DecValTok{2}
\FunctionTok{library}\NormalTok{(queueing)}
\NormalTok{t4p3 }\OtherTok{\textless{}{-}} \FunctionTok{NewInput.MMC}\NormalTok{(}\AttributeTok{lambda =}\NormalTok{ lambda, }\AttributeTok{mu =}\NormalTok{ mu, }\AttributeTok{c =}\NormalTok{ s, }\AttributeTok{n =} \DecValTok{40}\NormalTok{)}
\NormalTok{t4p3o }\OtherTok{\textless{}{-}} \FunctionTok{QueueingModel}\NormalTok{(t4p3)}
\FunctionTok{summary}\NormalTok{(t4p3o)}
\end{Highlighting}
\end{Shaded}

\begin{verbatim}
##      lambda mu c  k  m        RO        P0         Lq         Wq         X
## 1 0.6666667  2 2 NA NA 0.1666667 0.7142857 0.00952381 0.01428571 0.6666667
##           L         W Wqq Lqq
## 1 0.3428571 0.5142857 0.3 1.2
\end{verbatim}

\begin{itemize}
\tightlist
\item
  El tiempo en la cola es: \$ = 0.01428571\$ minutos = . El tiempo en el
  sistema: \$ W = 0.5142857 \$ Minutos = . El número de clientes en el
  sistema : \(L = 0.3428571\) y El número de clientes en la fila:
  \(L_q = 0.00952381\) *.
\end{itemize}

\hypertarget{ejercicio-3}{%
\subsection{Ejercicio 3}\label{ejercicio-3}}

\begin{verbatim}
Una gasolinera cuenta con una bomba de gasolina. Los automoviles que desean cargar llegan segun un proceso de Poisson a una tasa media de 15 por hora. Sin embargo, si la bomba esta en operacion, los clientes potenciales pueden desistir (ir a otra gasolinera). En particular, si hay n autos en ella, la probabilidad de que un cliente potencial que llega desista es n=3 para n = 1; 2; 3. El tiempo necesario para servir un auto tiene distribucion exponencial con media de 4 minutos.
\end{verbatim}

\(\lambda_0 = 15\) \(\lambda_1 = 15 - (15*1/3) = 10\) \$\lambda\_2 = 15
- (15*2/3) = 5 \$ \(\lambda_3 = 15 - (15*3/3) = 0\)
\(\bar\lambda = \sum_{n = 0}^\infty \lambda_n P_n = \lambda_0P_P + \lambda_1P_1 + \lambda_2P_2 + \lambda_3P_3 = 8.33\)

Calcular las \(P_n\) utilizando las ecuaciones de balance:

\(\mu*P_1 = \lambda_0*P_0\)
\(\lambda_0*P_0 + /mu*P_2 = (\lambda_1+\mu)*P_1\)
\(\lambda_1*P_1 + /mu*P_3 = (\lambda_2+\mu)*P_2\)
\(\lambda_2*P_2 + /mu*P_4 = (\lambda_3+\mu)*P_3\)

\begin{Shaded}
\begin{Highlighting}[]
\NormalTok{mu }\OtherTok{\textless{}{-}} \DecValTok{1}\SpecialCharTok{/}\DecValTok{15}
\NormalTok{lambda\_0 }\OtherTok{=} \DecValTok{15}
\NormalTok{lambda\_1 }\OtherTok{=} \DecValTok{10}
\NormalTok{lambda\_2 }\OtherTok{=} \DecValTok{5}
\NormalTok{lambda\_3 }\OtherTok{=} \DecValTok{0}
\end{Highlighting}
\end{Shaded}

\begin{Shaded}
\begin{Highlighting}[]
\NormalTok{lambdabarra }\OtherTok{\textless{}{-}} \FloatTok{8.33}
\end{Highlighting}
\end{Shaded}

a)Encuentre la distribución de probabilidad de estado estable del número
de autos en la gasolinera.

\$ L = \displaystyle\sum\_\{n=0\}\^{}\infty n P\_n = 0P\_0 + 1P\_1 +
2P\_2 + 3P\_3 = 4/3 = 1.33\$

\begin{Shaded}
\begin{Highlighting}[]
\NormalTok{L }\OtherTok{=} \FloatTok{1.33}
\end{Highlighting}
\end{Shaded}

\begin{enumerate}
\def\labelenumi{\alph{enumi})}
\setcounter{enumi}{1}
\tightlist
\item
  Encuentre el tiempo de espera esperado (incluido el servicio) de los
  automoviles que se quedan.
\end{enumerate}

\(W = L/\bar\lambda\)

\begin{Shaded}
\begin{Highlighting}[]
\NormalTok{W }\OtherTok{=}\NormalTok{ L}\SpecialCharTok{/}\NormalTok{lambdabarra }\CommentTok{\# 0.159639 hora = 9.57 minutos}
\end{Highlighting}
\end{Shaded}

\begin{itemize}
\item
  Sistema: Gasolinera, Clientes: autos, Servidores: 1 bomba de gasolina
  *
\item
  Tasa de llegadas (\(\lambda\)) = 15 autos/hora *
\item
  Tasa de servicios (\(\mu\)) = media 4 minutos = 1/15 hora *
\item
  Número de servidores = 1 *
\item
  Modelo: M/M/1 *
\end{itemize}

\begin{Shaded}
\begin{Highlighting}[]
\NormalTok{lambda }\OtherTok{\textless{}{-}} \DecValTok{15} \CommentTok{\#tasa de llegada}
\NormalTok{mu }\OtherTok{\textless{}{-}} \DecValTok{1}\SpecialCharTok{/}\DecValTok{15} \CommentTok{\# tasa de servicio}
\NormalTok{c }\OtherTok{\textless{}{-}} \DecValTok{1} 
\end{Highlighting}
\end{Shaded}

\hypertarget{ejercicio-4}{%
\subsection{Ejercicio 4}\label{ejercicio-4}}

Un departamento de una empresa tiene una operadora de procesador de
textos. Los documentos que se producen en el se entre-gan para ser
procesados de acuerdo con un proceso de Poisson con un tiempo esperado
entre llegadas de 30 min. Cuando la operadora tiene solo un documento
que procesar el tiempo esperado de servicio es de 20 minutos. Cuando hay
mas de un documento, la ayuda de edicion reduce este tiempo a 15 min. En
ambos casos, los tiempos de servicio tienen distribucion exponencial.

\begin{itemize}
\item
  Sistema: Departamento, Clientes: documentos, Servidores: operadora *
\item
  Tasa de llegadas (\(\lambda\)) = 1 documento cada 30 min *
\item
  Tasa de servicios (\(\mu\)) = 20 minutos solo 1 documento, 15 minutos
  mas de 1 documento *
\item
  Número de servidores = 1 *
\item
  Modelo: M/M/1 *
\end{itemize}

\begin{enumerate}
\def\labelenumi{\alph{enumi})}
\tightlist
\item
  Encuentre la distribucion de estado estable del numero de documentos
  que la operadora ya recibio pero todavia no procesa Calcular las
  \(P_n\) utilizando las ecuaciones de balance:
\end{enumerate}

\(\mu_1*P_1 = \lambda*P_0\)
\(\lambda*P_0 + /mu_2*P_2 = (\lambda+\mu_1)*P_1\)
\(\lambda*P_1 + /mu_2*P_3 = (\lambda+\mu_2)*P_2\)
\(\lambda*P_2 + /mu_2*P_4 = (\lambda+\mu_2)*P_3\) \(\dots\)

\(C_3 = \frac{\lambda\lambda\lambda}{\mu_1\mu_2\mu_3}\)

\begin{Shaded}
\begin{Highlighting}[]
\NormalTok{lambda }\OtherTok{\textless{}{-}} \DecValTok{2}\SpecialCharTok{/}\DecValTok{60} \CommentTok{\#2 por hora}
\NormalTok{mu1 }\OtherTok{\textless{}{-}} \DecValTok{3}\SpecialCharTok{/}\DecValTok{60} \CommentTok{\#20 minutos hasta 1 documento}
\NormalTok{mu2 }\OtherTok{\textless{}{-}} \DecValTok{4}\SpecialCharTok{/}\DecValTok{60} \CommentTok{\#15 minutos mas de 1 documento}

\NormalTok{c1 }\OtherTok{\textless{}{-}}\NormalTok{ lambda}\SpecialCharTok{/}\NormalTok{mu1}
\NormalTok{c2 }\OtherTok{\textless{}{-}}\NormalTok{ (lambda}\SpecialCharTok{*}\NormalTok{lambda)}\SpecialCharTok{/}\NormalTok{(mu1}\SpecialCharTok{*}\NormalTok{mu2)}
\NormalTok{c3 }\OtherTok{\textless{}{-}}\NormalTok{ (lambda}\SpecialCharTok{*}\NormalTok{lambda}\SpecialCharTok{*}\NormalTok{lambda)}\SpecialCharTok{/}\NormalTok{(mu1}\SpecialCharTok{*}\NormalTok{mu2}\SpecialCharTok{*}\NormalTok{mu2)}

\NormalTok{P0 }\OtherTok{\textless{}{-}} \DecValTok{1}\SpecialCharTok{/}\NormalTok{(c1}\SpecialCharTok{+}\NormalTok{c2}\SpecialCharTok{+}\NormalTok{c3)}
\NormalTok{P1 }\OtherTok{\textless{}{-}}\NormalTok{ c1}\SpecialCharTok{*}\NormalTok{P0}
\NormalTok{P2 }\OtherTok{\textless{}{-}}\NormalTok{ c2}\SpecialCharTok{*}\NormalTok{P0}
\NormalTok{P3 }\OtherTok{\textless{}{-}}\NormalTok{ c3}\SpecialCharTok{*}\NormalTok{P0}
\end{Highlighting}
\end{Shaded}

\begin{enumerate}
\def\labelenumi{\alph{enumi})}
\setcounter{enumi}{1}
\tightlist
\item
  Calcule L, Lq, W, Wq de este sistema.
\end{enumerate}

\begin{Shaded}
\begin{Highlighting}[]
\NormalTok{Pn }\OtherTok{\textless{}{-}} \FunctionTok{c}\NormalTok{(P0,P1,P2,P3)}
\NormalTok{n }\OtherTok{\textless{}{-}} \FunctionTok{c}\NormalTok{(}\DecValTok{0}\NormalTok{,}\DecValTok{1}\NormalTok{,}\DecValTok{2}\NormalTok{,}\DecValTok{3}\NormalTok{)}
\NormalTok{L }\OtherTok{=} \FunctionTok{sum}\NormalTok{(n}\SpecialCharTok{*}\NormalTok{Pn)}
\FunctionTok{print}\NormalTok{(L) }\CommentTok{\# el numero de documentos promedio que llegan al sistema}
\end{Highlighting}
\end{Shaded}

\begin{verbatim}
## [1] 1.571429
\end{verbatim}

\begin{Shaded}
\begin{Highlighting}[]
\NormalTok{s }\OtherTok{\textless{}{-}} \DecValTok{1}
\NormalTok{n }\OtherTok{\textless{}{-}} \FunctionTok{c}\NormalTok{(}\DecValTok{1}\NormalTok{,}\DecValTok{2}\NormalTok{,}\DecValTok{3}\NormalTok{,}\DecValTok{4}\NormalTok{)}
\NormalTok{Lq }\OtherTok{=} \FunctionTok{sum}\NormalTok{((n}\SpecialCharTok{{-}}\NormalTok{s)}\SpecialCharTok{*}\NormalTok{Pn) }\CommentTok{\# el numero de documentos en la fila, esperando a ser procesados}
\FunctionTok{print}\NormalTok{(Lq)}
\end{Highlighting}
\end{Shaded}

\begin{verbatim}
## [1] 1.571429
\end{verbatim}

\begin{Shaded}
\begin{Highlighting}[]
\NormalTok{W }\OtherTok{=}\NormalTok{ L}\SpecialCharTok{/}\NormalTok{lambda}
\FunctionTok{print}\NormalTok{(W)}\CommentTok{\# 2 dias, tiempo en el que se va a procesar el documento}
\end{Highlighting}
\end{Shaded}

\begin{verbatim}
## [1] 47.14286
\end{verbatim}

\begin{Shaded}
\begin{Highlighting}[]
\NormalTok{Wq }\OtherTok{=}\NormalTok{  Lq}\SpecialCharTok{/}\NormalTok{lambda}
\FunctionTok{print}\NormalTok{(Wq) }\CommentTok{\#tiempo que el documento va a esperar a ser procesado}
\end{Highlighting}
\end{Shaded}

\begin{verbatim}
## [1] 47.14286
\end{verbatim}

\hypertarget{ejercicio-5}{%
\subsection{Ejercicio 5}\label{ejercicio-5}}

\begin{itemize}
\item
  Sistema: peluquería, Clientes: personas, Servidores: peluquero *
\item
  Tasa de llegadas (\(\lambda\)) = 4 clientes por hora = 4/60*
\item
  Tasa de servicios (\(\mu\)) = 15 minutos por cliente = 4/60 *
\item
  Número de servidores = 1 *
\item
  Tamaño de la cola = 3 *
\item
  Modelo: M/M/1/K *
\end{itemize}

\begin{Shaded}
\begin{Highlighting}[]
\NormalTok{mu }\OtherTok{=} \DecValTok{4}\SpecialCharTok{/}\DecValTok{60}
\NormalTok{lambda }\OtherTok{=} \DecValTok{4}\SpecialCharTok{/}\DecValTok{60}
\NormalTok{s }\OtherTok{\textless{}{-}} \DecValTok{1}

\FunctionTok{library}\NormalTok{(queueing)}
\NormalTok{t4p5 }\OtherTok{\textless{}{-}} \FunctionTok{NewInput.MM1K}\NormalTok{(}\AttributeTok{lambda =}\NormalTok{ lambda, }\AttributeTok{mu =}\NormalTok{ mu, }\AttributeTok{k =} \DecValTok{4}\NormalTok{)}
\FunctionTok{CheckInput}\NormalTok{(t4p5)}
\NormalTok{t4p5o }\OtherTok{=} \FunctionTok{QueueingModel}\NormalTok{(t4p5)}
\FunctionTok{summary}\NormalTok{(t4p5o)}
\end{Highlighting}
\end{Shaded}

\begin{verbatim}
##       lambda         mu c k  m  RO  P0  Lq   Wq          X L    W Wqq Lqq
## 1 0.06666667 0.06666667 1 4 NA 0.8 0.2 1.2 22.5 0.05333333 2 37.5  30   2
\end{verbatim}

\begin{enumerate}
\def\labelenumi{\alph{enumi})}
\tightlist
\item
  Las probabilidades
\end{enumerate}

\begin{Shaded}
\begin{Highlighting}[]
\FunctionTok{Pn}\NormalTok{(t4p5o)}
\end{Highlighting}
\end{Shaded}

\begin{verbatim}
## [1] 0.2 0.2 0.2 0.2 0.2
\end{verbatim}

\begin{enumerate}
\def\labelenumi{\alph{enumi})}
\setcounter{enumi}{1}
\tightlist
\item
  La cantidad esperada de clientes en la peluquería.
\end{enumerate}

\begin{Shaded}
\begin{Highlighting}[]
\FunctionTok{L}\NormalTok{(t4p5o) }\CommentTok{\#La cantidad de clientes esperada es 2}
\end{Highlighting}
\end{Shaded}

\begin{verbatim}
## [1] 2
\end{verbatim}

\begin{enumerate}
\def\labelenumi{\alph{enumi})}
\setcounter{enumi}{2}
\tightlist
\item
  la probabilidad de que los clientes se vayan a otra parte porque el
  local está lleno.
\end{enumerate}

\begin{itemize}
\tightlist
\item
  La probabilidad de que los clientes se vayan a otra parte porque el
  local está lleno es de 0.2 *
\end{itemize}

\hypertarget{ejercicio-6}{%
\section{Ejercicio 6}\label{ejercicio-6}}

Juan Salas es alumno en la UAdeC. Hace trabajos extras para aumentar sus
ingresos. Las peticiones de trabajo llegan en promedio cada 5 días, pero
el tiempo entre ellas es exponencial. El tiempo para terminar un trabajo
también es exponencial, con una media de 4 días.

Sistema: Trabajos extras que hace Juan Salas, Clientes: Trabajos,
Servidores: Juan Salas

Tasa de llegadas (\(\lambda\)) = 1 trabajo cada 5 dias = 1.4 trabajos
cada semana

Tasa de servicio (\(\mu\)) = 1 trabajo para 4 días = 1.75 trabajos en
una semana

Numero de servidores = 1

Modelo: M/M/1/K

\begin{enumerate}
\def\labelenumi{\alph{enumi})}
\tightlist
\item
  ¿Cúal es la probabilidad de que le falte trabajo a Juan?
\end{enumerate}

\begin{Shaded}
\begin{Highlighting}[]
\NormalTok{n}\OtherTok{=}\DecValTok{0}
\NormalTok{lambda}\OtherTok{=} \FloatTok{1.4}
\NormalTok{mu}\OtherTok{=} \FloatTok{1.75}
\NormalTok{t4p6}\OtherTok{=} \FunctionTok{NewInput.MM1}\NormalTok{(}\AttributeTok{lambda=}\NormalTok{ lambda, }\AttributeTok{mu=}\NormalTok{ mu, }\AttributeTok{n=}\NormalTok{ n)}
\FunctionTok{CheckInput}\NormalTok{(t4p6)}
\NormalTok{t4p6o}\OtherTok{=} \FunctionTok{QueueingModel}\NormalTok{(t4p6)}
\FunctionTok{summary}\NormalTok{(t4p6o)}
\end{Highlighting}
\end{Shaded}

\begin{verbatim}
##   lambda   mu c  k  m  RO  P0  Lq       Wq   X L        W      Wqq Lqq
## 1    1.4 1.75 1 NA NA 0.8 0.2 3.2 2.285714 1.4 4 2.857143 2.857143   5
\end{verbatim}

La probabilidad de que le falte trabajo es de 0.2

\begin{enumerate}
\def\labelenumi{\alph{enumi})}
\setcounter{enumi}{1}
\tightlist
\item
  Si Juan cobra unos \$50 por cada trabajo, ¿cuál es su ingreso mensual
  promedio?
\end{enumerate}

\begin{Shaded}
\begin{Highlighting}[]
\NormalTok{mensual }\OtherTok{=} \DecValTok{30}\SpecialCharTok{/}\DecValTok{5} \CommentTok{\#6 trabajos cada mes}
\NormalTok{ingmensual}\OtherTok{=} \DecValTok{50}\SpecialCharTok{*}\NormalTok{mensual}
\FunctionTok{print}\NormalTok{(ingmensual)}
\end{Highlighting}
\end{Shaded}

\begin{verbatim}
## [1] 300
\end{verbatim}

\begin{Shaded}
\begin{Highlighting}[]
\FunctionTok{Lq}\NormalTok{(t4p6o)}
\end{Highlighting}
\end{Shaded}

\begin{verbatim}
## [1] 3.2
\end{verbatim}

su ingreso mensual promedio es de \$300

\begin{enumerate}
\def\labelenumi{\alph{enumi})}
\setcounter{enumi}{2}
\tightlist
\item
  Si al final del semestre Juan decide subcontratar los trabajos
  pendientes a \$40 cada uno, ¿cuánto debe esperar pagar en promedio?
\end{enumerate}

\begin{Shaded}
\begin{Highlighting}[]
\NormalTok{pendientes}\OtherTok{=}\FunctionTok{Lq}\NormalTok{(t4p6o)}
\NormalTok{pagar}\OtherTok{=}\DecValTok{40}\SpecialCharTok{*}\NormalTok{pendientes}
\FunctionTok{print}\NormalTok{(pagar)}
\end{Highlighting}
\end{Shaded}

\begin{verbatim}
## [1] 128
\end{verbatim}

debe esperar pagar en promedio es \$128

\hypertarget{ejercicio-7}{%
\section{Ejercicio 7}\label{ejercicio-7}}

Las probabilidades de que hayan clientes en un sistema (M/M/1) :
\((DG/5/∞)\) se ven en la tabla siguiente:

\begin{longtable}[]{@{}llllll@{}}
\toprule
n & 0 & 1 & 2 & 3 & 4 \\
\midrule
\endhead
Pn & 0.399 & 0.249 & 0.156 & 0.097 & 0.061 \\
\bottomrule
\end{longtable}

La frecuencia de llegada \(λ\) es 5 clientes por hora. La rapidez del
servicio \(μ\) es 8 clientes por hora. Calcule lo siguiente:

\begin{Shaded}
\begin{Highlighting}[]
\NormalTok{n0}\OtherTok{=}\DecValTok{0}
\NormalTok{n1}\OtherTok{=}\DecValTok{1}
\NormalTok{n2}\OtherTok{=}\DecValTok{2}
\NormalTok{n3}\OtherTok{=}\DecValTok{3}
\NormalTok{n4}\OtherTok{=}\DecValTok{4}
\NormalTok{n5}\OtherTok{=}\DecValTok{5}
\NormalTok{P0}\OtherTok{=}\FloatTok{0.399}
\NormalTok{P1}\OtherTok{=}\FloatTok{0.249}
\NormalTok{P2}\OtherTok{=}\FloatTok{0.156}
\NormalTok{P3}\OtherTok{=}\FloatTok{0.097}
\NormalTok{P4}\OtherTok{=}\FloatTok{0.061}
\NormalTok{P5}\OtherTok{=}\FloatTok{0.038}
\NormalTok{Sumatoria}\OtherTok{=}\FunctionTok{sum}\NormalTok{(P0,P1,P2,P3,P4)}
\FunctionTok{print}\NormalTok{(Sumatoria)}
\end{Highlighting}
\end{Shaded}

\begin{verbatim}
## [1] 0.962
\end{verbatim}

\begin{enumerate}
\def\labelenumi{\alph{enumi})}
\tightlist
\item
  La probabilidad de que un cliente que llega pueda entrar al sistema.
\end{enumerate}

\begin{Shaded}
\begin{Highlighting}[]
\FunctionTok{print}\NormalTok{(Sumatoria)}
\end{Highlighting}
\end{Shaded}

\begin{verbatim}
## [1] 0.962
\end{verbatim}

La probabilidad de que un cliente que llega pueda entrar al sistema es
de 0.962.

\begin{enumerate}
\def\labelenumi{\alph{enumi})}
\setcounter{enumi}{1}
\tightlist
\item
  La frecuencia con la que los clientes que llegan no pueden entrar al
  sistema.
\end{enumerate}

\begin{Shaded}
\begin{Highlighting}[]
\NormalTok{frec}\OtherTok{=} \DecValTok{5}\SpecialCharTok{*}\FloatTok{0.038}
\FunctionTok{print}\NormalTok{(frec)}
\end{Highlighting}
\end{Shaded}

\begin{verbatim}
## [1] 0.19
\end{verbatim}

La frecuencia con la que los clientes que llegan no pueden entrar al
sistema es de 0.19 clientes por hora.

\begin{enumerate}
\def\labelenumi{\alph{enumi})}
\setcounter{enumi}{2}
\tightlist
\item
  La cantidad esperada de clientes en el sistema.
\end{enumerate}

\begin{Shaded}
\begin{Highlighting}[]
\NormalTok{n0}\OtherTok{=}\DecValTok{0}
\NormalTok{n1}\OtherTok{=}\DecValTok{1}
\NormalTok{n2}\OtherTok{=}\DecValTok{2}
\NormalTok{n3}\OtherTok{=}\DecValTok{3}
\NormalTok{n4}\OtherTok{=}\DecValTok{4}
\NormalTok{n5}\OtherTok{=}\DecValTok{5}
\NormalTok{P0}\OtherTok{=}\FloatTok{0.399}\SpecialCharTok{*}\NormalTok{n0}
\NormalTok{P1}\OtherTok{=}\FloatTok{0.249}\SpecialCharTok{*}\NormalTok{n1}
\NormalTok{P2}\OtherTok{=}\FloatTok{0.156}\SpecialCharTok{*}\NormalTok{n2}
\NormalTok{P3}\OtherTok{=}\FloatTok{0.097}\SpecialCharTok{*}\NormalTok{n3}
\NormalTok{P4}\OtherTok{=}\FloatTok{0.061}\SpecialCharTok{*}\NormalTok{n4}
\NormalTok{P5}\OtherTok{=}\FloatTok{0.038}\SpecialCharTok{*}\NormalTok{n5}
\NormalTok{Ls}\OtherTok{=}\FunctionTok{sum}\NormalTok{(P0,P1,P2,P3,P4,P5)}
\FunctionTok{print}\NormalTok{(Ls)}
\end{Highlighting}
\end{Shaded}

\begin{verbatim}
## [1] 1.286
\end{verbatim}

d)El tiempo promedio de espera en la cola.

\begin{Shaded}
\begin{Highlighting}[]
\NormalTok{Lambda7}\OtherTok{\textless{}{-}}\NormalTok{(}\DecValTok{5}\SpecialCharTok{*}\NormalTok{(}\DecValTok{1}\FloatTok{{-}0.038}\NormalTok{))}
\FunctionTok{print}\NormalTok{(Lambda7)}
\end{Highlighting}
\end{Shaded}

\begin{verbatim}
## [1] 4.81
\end{verbatim}

\begin{Shaded}
\begin{Highlighting}[]
\NormalTok{Ws}\OtherTok{\textless{}{-}}\NormalTok{(Ls}\SpecialCharTok{/}\NormalTok{Lambda7)}
\FunctionTok{print}\NormalTok{(Ws)}
\end{Highlighting}
\end{Shaded}

\begin{verbatim}
## [1] 0.2673597
\end{verbatim}

\begin{Shaded}
\begin{Highlighting}[]
\NormalTok{Wq}\OtherTok{\textless{}{-}}\NormalTok{(Ws}\SpecialCharTok{{-}}\NormalTok{(}\DecValTok{1}\SpecialCharTok{/}\DecValTok{8}\NormalTok{))}
\FunctionTok{print}\NormalTok{(Wq)}
\end{Highlighting}
\end{Shaded}

\begin{verbatim}
## [1] 0.1423597
\end{verbatim}

\hypertarget{ejercicio-8}{%
\subsection{Ejercicio 8}\label{ejercicio-8}}

El centro de computo de la UAdeC tiene cuatro computadoras principales
identicas. La cantidad de usuarios en cualquier momento es de 25. Cada
usuario puede solicitar un trabajo por una terminal, cada 15 minutos en
promedio, pero el tiempo real entre solicitudes es exponencial. Los
trabajos que llegan pasan en forma automatica a la primera computadora
disponible. El tiempo de ejecucion por solicitud es exponencial, con un
promedio de 2 minutos. Calcule lo siguiente:

\begin{itemize}
\item
  Los clientes: Los usuarios. *
\item
  El sistema: El centro de cómputo. *
\item
  Los servidores: Las computadoras. *
\item
  Tasa de llegadas (\(\lambda\)): 1 trabajo cada 15 minutos *
\item
  Tasa de servicio (\(\mu\)): 1 trabajo cada 2 minutos = 30
  trabajos/hora. *
\item
  Número de servidores: 4 *
\item
  Tamaño de la cola: 25 *
\item
  Tipo de cola: M/M/S/K/M * Calcule lo siguiente:
\end{itemize}

\begin{Shaded}
\begin{Highlighting}[]
\NormalTok{mu }\OtherTok{=} \DecValTok{4}\SpecialCharTok{/}\DecValTok{60}
\NormalTok{lambda }\OtherTok{=} \DecValTok{30}\SpecialCharTok{/}\DecValTok{60}
\NormalTok{s }\OtherTok{\textless{}{-}} \DecValTok{4}
\end{Highlighting}
\end{Shaded}

\begin{Shaded}
\begin{Highlighting}[]
\NormalTok{t4p8a }\OtherTok{\textless{}{-}} \FunctionTok{NewInput.MMCK}\NormalTok{(}\AttributeTok{lambda =}\NormalTok{ lambda, }\AttributeTok{mu =}\NormalTok{ mu,}\AttributeTok{c =}\NormalTok{ s, }\AttributeTok{k =} \DecValTok{25}\NormalTok{)}
\FunctionTok{CheckInput}\NormalTok{(t4p8a)}
\NormalTok{t4p8ao }\OtherTok{\textless{}{-}} \FunctionTok{QueueingModel}\NormalTok{(t4p8a)}
\end{Highlighting}
\end{Shaded}

\begin{verbatim}
## Warning in formals(fun): argument is not a function

## Warning in formals(fun): argument is not a function

## Warning in formals(fun): argument is not a function

## Warning in formals(fun): argument is not a function

## Warning in formals(fun): argument is not a function

## Warning in formals(fun): argument is not a function

## Warning in formals(fun): argument is not a function

## Warning in formals(fun): argument is not a function

## Warning in formals(fun): argument is not a function

## Warning in formals(fun): argument is not a function

## Warning in formals(fun): argument is not a function

## Warning in formals(fun): argument is not a function

## Warning in formals(fun): argument is not a function

## Warning in formals(fun): argument is not a function

## Warning in formals(fun): argument is not a function

## Warning in formals(fun): argument is not a function

## Warning in formals(fun): argument is not a function
\end{verbatim}

\begin{Shaded}
\begin{Highlighting}[]
\NormalTok{t4p8b }\OtherTok{\textless{}{-}} \FunctionTok{NewInput.MMCKK}\NormalTok{(}\AttributeTok{lambda =}\NormalTok{ lambda, }\AttributeTok{mu =}\NormalTok{ mu,}\AttributeTok{c =}\NormalTok{ s, }\AttributeTok{k =} \DecValTok{25}\NormalTok{)}
\FunctionTok{CheckInput}\NormalTok{(t4p8b)}
\NormalTok{t4p8bo }\OtherTok{\textless{}{-}} \FunctionTok{QueueingModel}\NormalTok{(t4p8b)}
\end{Highlighting}
\end{Shaded}

\begin{verbatim}
## Warning in formals(fun): argument is not a function

## Warning in formals(fun): argument is not a function

## Warning in formals(fun): argument is not a function

## Warning in formals(fun): argument is not a function

## Warning in formals(fun): argument is not a function

## Warning in formals(fun): argument is not a function

## Warning in formals(fun): argument is not a function

## Warning in formals(fun): argument is not a function

## Warning in formals(fun): argument is not a function

## Warning in formals(fun): argument is not a function

## Warning in formals(fun): argument is not a function

## Warning in formals(fun): argument is not a function

## Warning in formals(fun): argument is not a function

## Warning in formals(fun): argument is not a function

## Warning in formals(fun): argument is not a function

## Warning in formals(fun): argument is not a function

## Warning in formals(fun): argument is not a function
\end{verbatim}

\begin{Shaded}
\begin{Highlighting}[]
\FunctionTok{CompareQueueingModels}\NormalTok{(t4p8ao,t4p8bo)}
\end{Highlighting}
\end{Shaded}

\begin{verbatim}
##   lambda         mu c  k  m        RO           P0       Lq       Wq         X
## 1    0.5 0.06666667 4 25 NA 0.9999997 6.545554e-09 19.85715 74.46433 0.2666666
## 2    0.5 0.06666667 4 25 NA 1.0000000 5.304813e-34 20.46667 76.75000 0.2666667
##          L        W      Wqq      Lqq
## 1 23.85715 89.46433 74.46443 19.85718
## 2 24.46667 91.75000 76.75000 20.46667
\end{verbatim}

\begin{Shaded}
\begin{Highlighting}[]
\NormalTok{t4p8c }\OtherTok{\textless{}{-}} \FunctionTok{NewInput.MMCKM}\NormalTok{(}\AttributeTok{lambda =}\NormalTok{ lambda, }\AttributeTok{mu =}\NormalTok{ mu, }\AttributeTok{k =}\NormalTok{ s, }\AttributeTok{m =} \DecValTok{25}\NormalTok{)}
\FunctionTok{CheckInput}\NormalTok{(t4p8c)}
\NormalTok{t4p8co }\OtherTok{\textless{}{-}} \FunctionTok{QueueingModel}\NormalTok{(t4p8c)}
\end{Highlighting}
\end{Shaded}

\begin{Shaded}
\begin{Highlighting}[]
\FunctionTok{summary}\NormalTok{(t4p8co)}
\end{Highlighting}
\end{Shaded}

\begin{verbatim}
##   lambda         mu c k  m RO           P0       Lq       Wq          X
## 1    0.5 0.06666667 1 4 25  1 1.034698e-09 2.993906 44.90859 0.06666667
##          L        W      Wqq      Lqq
## 1 3.993906 59.90859 44.90859 2.993906
\end{verbatim}

\begin{enumerate}
\def\labelenumi{\alph{enumi})}
\tightlist
\item
  La probabilidad de que un trabajo no se ejecute de inmediato al
  solicitarlo.
\end{enumerate}

\begin{Shaded}
\begin{Highlighting}[]
\CommentTok{\# Hay que calcular la probabilidad }
\FunctionTok{Pn}\NormalTok{(t4p8co)}
\end{Highlighting}
\end{Shaded}

\begin{verbatim}
## [1] 1.034698e-09 1.940060e-07 3.492107e-05 6.023885e-03 9.939410e-01
\end{verbatim}

\begin{enumerate}
\def\labelenumi{\alph{enumi})}
\setcounter{enumi}{1}
\tightlist
\item
  El tiempo promedio en el que el usuario obtiene sus resultados.
\end{enumerate}

\begin{Shaded}
\begin{Highlighting}[]
\FunctionTok{W}\NormalTok{(t4p8co)}
\end{Highlighting}
\end{Shaded}

\begin{verbatim}
## [1] 59.90859
\end{verbatim}

\begin{itemize}
\tightlist
\item
  El tiempo promedio en el que el usuario obtiene su resultados es de 15
  horas. *
\end{itemize}

\begin{enumerate}
\def\labelenumi{\alph{enumi})}
\setcounter{enumi}{2}
\tightlist
\item
  La cantidad promedio de trabajos que esperan su procesamiento.
\end{enumerate}

\begin{Shaded}
\begin{Highlighting}[]
\FunctionTok{Lq}\NormalTok{(t4p8co)}
\end{Highlighting}
\end{Shaded}

\begin{verbatim}
## [1] 2.993906
\end{verbatim}

\begin{enumerate}
\def\labelenumi{\alph{enumi})}
\setcounter{enumi}{3}
\tightlist
\item
  El porcentaje del tiempo durante el cual el centro de cómputo está
  inactivo.
\end{enumerate}

\begin{Shaded}
\begin{Highlighting}[]
\CommentTok{\#Calculamos la P0 (probabilidad de que haya cero clientes en el sistema)}
\end{Highlighting}
\end{Shaded}

\begin{itemize}
\tightlist
\item
  El porcentaje del tiempo durante el cual el centro de cómputo está
  inactivo es 2.438265e-06\%. *
\end{itemize}

\begin{enumerate}
\def\labelenumi{\alph{enumi})}
\setcounter{enumi}{4}
\tightlist
\item
  La cantidad promedio de computadoras ociosas.
\end{enumerate}

\begin{Shaded}
\begin{Highlighting}[]
\FunctionTok{L}\NormalTok{(t4p8co) }\SpecialCharTok{{-}} \FunctionTok{Lq}\NormalTok{(t4p8co)}
\end{Highlighting}
\end{Shaded}

\begin{verbatim}
## [1] 1
\end{verbatim}

\begin{itemize}
\tightlist
\item
  El promedio todas las computadorads estarán ocupadas, por lo que no
  habrá computadoras ociosas. *
\end{itemize}

\hypertarget{ejercicio-9}{%
\subsection{Ejercicio 9}\label{ejercicio-9}}

Eat \& Gas es una gasolinera con dos bombas. El carril que llega a ellas
puede dar cabida cuando mucho a cinco automóviles, incluyendo los que
llenan el tanque. Los que llegan cuando el carril está lleno van a otra
parte. La distribución de los vehículos que llegan es de Poisson, con
promedio de 20 por hora.

El tiempo para llenar y pagar las compras es exponencial, con 6 minutos
de promedio. \(\lambda=20\) vehiculos/hora \(\mu=1/6 * 60/1=10\)
vehiculos/hora \(p=20/10=2\) \(N=5\) \(c=2\) \(p/c=2/2=1\)

\begin{enumerate}
\def\labelenumi{\alph{enumi})}
\tightlist
\item
  El porcentaje de automóviles que llenarán el tanque en otro lado.
\end{enumerate}

\begin{Shaded}
\begin{Highlighting}[]
\NormalTok{lambda}\OtherTok{=}\DecValTok{20}
\NormalTok{mu}\OtherTok{=}\DecValTok{10}
\NormalTok{N}\OtherTok{=}\DecValTok{5}
\NormalTok{s}\OtherTok{=}\DecValTok{2}
\NormalTok{p0}\OtherTok{=}\DecValTok{1}\SpecialCharTok{/}\NormalTok{((lambda}\SpecialCharTok{/}\NormalTok{mu)}\SpecialCharTok{\^{}}\DecValTok{0}\SpecialCharTok{/}\NormalTok{(}\DecValTok{1}\NormalTok{)}\SpecialCharTok{+}\NormalTok{(lambda}\SpecialCharTok{/}\NormalTok{mu)}\SpecialCharTok{\^{}}\DecValTok{1}\SpecialCharTok{/}\NormalTok{(}\DecValTok{1}\NormalTok{)}\SpecialCharTok{+}\NormalTok{((lambda}\SpecialCharTok{/}\NormalTok{mu)}\SpecialCharTok{\^{}}\DecValTok{2}\SpecialCharTok{/}\NormalTok{(}\DecValTok{2}\NormalTok{))}\SpecialCharTok{*}\NormalTok{(}\DecValTok{1}\SpecialCharTok{/}\DecValTok{1}\SpecialCharTok{{-}}\NormalTok{(lambda}\SpecialCharTok{/}\NormalTok{(s}\SpecialCharTok{*}\NormalTok{mu))))}
\NormalTok{p5}\OtherTok{=}\NormalTok{p0}\SpecialCharTok{*}\NormalTok{(((lambda}\SpecialCharTok{/}\NormalTok{mu)}\SpecialCharTok{\^{}}\DecValTok{5}\NormalTok{)}\SpecialCharTok{/}\NormalTok{(}\DecValTok{2}\SpecialCharTok{*}\DecValTok{2}\SpecialCharTok{\^{}}\NormalTok{(}\DecValTok{5{-}2}\NormalTok{)))}

\FunctionTok{print}\NormalTok{(p5)}
\end{Highlighting}
\end{Shaded}

\begin{verbatim}
## [1] 0.6666667
\end{verbatim}

\textbf{El 66.66\% de los vrhiculos llenan su taquen en otro lado}

\begin{enumerate}
\def\labelenumi{\alph{enumi})}
\setcounter{enumi}{1}
\tightlist
\item
  El porcentaje de tiempo en el que se usa una bomba.
\end{enumerate}

\begin{Shaded}
\begin{Highlighting}[]
\NormalTok{lambda}\OtherTok{=}\DecValTok{20}
\NormalTok{mu}\OtherTok{=}\DecValTok{10}
\NormalTok{N}\OtherTok{=}\DecValTok{5}
\NormalTok{s}\OtherTok{=}\DecValTok{2}
\NormalTok{p0}\OtherTok{=}\DecValTok{1}\SpecialCharTok{/}\NormalTok{((lambda}\SpecialCharTok{/}\NormalTok{mu)}\SpecialCharTok{\^{}}\DecValTok{0}\SpecialCharTok{/}\NormalTok{(}\DecValTok{1}\NormalTok{)}\SpecialCharTok{+}\NormalTok{(lambda}\SpecialCharTok{/}\NormalTok{mu)}\SpecialCharTok{\^{}}\DecValTok{1}\SpecialCharTok{/}\NormalTok{(}\DecValTok{1}\NormalTok{)}\SpecialCharTok{+}\NormalTok{((lambda}\SpecialCharTok{/}\NormalTok{mu)}\SpecialCharTok{\^{}}\DecValTok{2}\SpecialCharTok{/}\NormalTok{(}\DecValTok{2}\NormalTok{))}\SpecialCharTok{*}\NormalTok{(}\DecValTok{1}\SpecialCharTok{/}\DecValTok{1}\SpecialCharTok{{-}}\NormalTok{(lambda}\SpecialCharTok{/}\NormalTok{(s}\SpecialCharTok{*}\NormalTok{mu))))}
\NormalTok{p1}\OtherTok{=}\NormalTok{p0}\SpecialCharTok{*}\NormalTok{(}\DecValTok{8}\SpecialCharTok{/}\DecValTok{3}\NormalTok{)}\SpecialCharTok{\^{}}\DecValTok{1}\SpecialCharTok{/}\DecValTok{1}
\FunctionTok{print}\NormalTok{(p1)}
\end{Highlighting}
\end{Shaded}

\begin{verbatim}
## [1] 0.8888889
\end{verbatim}

\textbf{Se utiliza una bomba el 88.8\%} c) La utilización porcentual de
las dos bombas.

\begin{Shaded}
\begin{Highlighting}[]
\NormalTok{lambda}\OtherTok{=}\DecValTok{20}
\NormalTok{mu}\OtherTok{=}\DecValTok{10}
\NormalTok{N}\OtherTok{=}\DecValTok{5}
\NormalTok{s}\OtherTok{=}\DecValTok{2}
\NormalTok{p0}\OtherTok{=}\DecValTok{1}\SpecialCharTok{/}\NormalTok{((lambda}\SpecialCharTok{/}\NormalTok{mu)}\SpecialCharTok{\^{}}\DecValTok{0}\SpecialCharTok{/}\NormalTok{(}\DecValTok{1}\NormalTok{)}\SpecialCharTok{+}\NormalTok{(lambda}\SpecialCharTok{/}\NormalTok{mu)}\SpecialCharTok{\^{}}\DecValTok{1}\SpecialCharTok{/}\NormalTok{(}\DecValTok{1}\NormalTok{)}\SpecialCharTok{+}\NormalTok{((lambda}\SpecialCharTok{/}\NormalTok{mu)}\SpecialCharTok{\^{}}\DecValTok{2}\SpecialCharTok{/}\NormalTok{(}\DecValTok{2}\NormalTok{))}\SpecialCharTok{*}\NormalTok{(}\DecValTok{1}\SpecialCharTok{/}\DecValTok{1}\SpecialCharTok{{-}}\NormalTok{(lambda}\SpecialCharTok{/}\NormalTok{(s}\SpecialCharTok{*}\NormalTok{mu))))}
\NormalTok{p5}\OtherTok{=}\NormalTok{p0}\SpecialCharTok{*}\NormalTok{(((lambda}\SpecialCharTok{/}\NormalTok{mu)}\SpecialCharTok{\^{}}\DecValTok{5}\NormalTok{)}\SpecialCharTok{/}\NormalTok{(}\DecValTok{2}\SpecialCharTok{*}\DecValTok{2}\SpecialCharTok{\^{}}\NormalTok{(}\DecValTok{5{-}2}\NormalTok{)))}
\NormalTok{c1}\OtherTok{=}\NormalTok{((}\DecValTok{1}\SpecialCharTok{{-}}\NormalTok{p5)}\SpecialCharTok{/}\FloatTok{7.5}\NormalTok{)}\SpecialCharTok{*}\NormalTok{lambda}
\NormalTok{ct}\OtherTok{=}\NormalTok{c1}\SpecialCharTok{/}\NormalTok{s}
\FunctionTok{print}\NormalTok{(ct)}
\end{Highlighting}
\end{Shaded}

\begin{verbatim}
## [1] 0.4444444
\end{verbatim}

\textbf{El uso porcentual es 0.4444}

\begin{enumerate}
\def\labelenumi{\alph{enumi})}
\setcounter{enumi}{3}
\tightlist
\item
  La probabilidad de que un automóvil que llegue no reciba servicio de
  inmediato, sino que se forme en la cola.
\end{enumerate}

\begin{Shaded}
\begin{Highlighting}[]
\NormalTok{lambda}\OtherTok{=}\DecValTok{20}
\NormalTok{mu}\OtherTok{=}\DecValTok{10}
\NormalTok{N}\OtherTok{=}\DecValTok{5}
\NormalTok{s}\OtherTok{=}\DecValTok{2}
\NormalTok{p0}\OtherTok{=}\DecValTok{1}\SpecialCharTok{/}\NormalTok{((lambda}\SpecialCharTok{/}\NormalTok{mu)}\SpecialCharTok{\^{}}\DecValTok{0}\SpecialCharTok{/}\NormalTok{(}\DecValTok{1}\NormalTok{)}\SpecialCharTok{+}\NormalTok{(lambda}\SpecialCharTok{/}\NormalTok{mu)}\SpecialCharTok{\^{}}\DecValTok{1}\SpecialCharTok{/}\NormalTok{(}\DecValTok{1}\NormalTok{)}\SpecialCharTok{+}\NormalTok{((lambda}\SpecialCharTok{/}\NormalTok{mu)}\SpecialCharTok{\^{}}\DecValTok{2}\SpecialCharTok{/}\NormalTok{(}\DecValTok{2}\NormalTok{))}\SpecialCharTok{*}\NormalTok{(}\DecValTok{1}\SpecialCharTok{/}\DecValTok{1}\SpecialCharTok{{-}}\NormalTok{(lambda}\SpecialCharTok{/}\NormalTok{(s}\SpecialCharTok{*}\NormalTok{mu))))}
\NormalTok{p1}\OtherTok{=}\NormalTok{p0}\SpecialCharTok{*}\NormalTok{(}\DecValTok{8}\SpecialCharTok{/}\DecValTok{3}\NormalTok{)}\SpecialCharTok{\^{}}\DecValTok{1}\SpecialCharTok{/}\DecValTok{1}
\NormalTok{pn}\OtherTok{=}\FunctionTok{abs}\NormalTok{(}\DecValTok{1}\SpecialCharTok{{-}}\NormalTok{p0}\SpecialCharTok{{-}}\NormalTok{p1)}
\FunctionTok{print}\NormalTok{(pn)}
\end{Highlighting}
\end{Shaded}

\begin{verbatim}
## [1] 0.2222222
\end{verbatim}

\textbf{La probabilidad de no recibir servicio al llegar es de 22\%}

\begin{enumerate}
\def\labelenumi{\alph{enumi})}
\setcounter{enumi}{4}
\tightlist
\item
  La capacidad del carril que asegure que, en promedio, no haya más del
  10\% de los vehículos que llegan se vayan a otra parte. estén
  inactivas sea 0.05 o menos.
\end{enumerate}

\begin{Shaded}
\begin{Highlighting}[]
\NormalTok{lambda}\OtherTok{=}\DecValTok{20}
\NormalTok{mu}\OtherTok{=}\DecValTok{10}
\NormalTok{N}\OtherTok{=}\DecValTok{5}
\NormalTok{s}\OtherTok{=}\DecValTok{2}
\NormalTok{p0}\OtherTok{=}\DecValTok{1}\SpecialCharTok{/}\NormalTok{((lambda}\SpecialCharTok{/}\NormalTok{mu)}\SpecialCharTok{\^{}}\DecValTok{0}\SpecialCharTok{/}\NormalTok{(}\DecValTok{1}\NormalTok{)}\SpecialCharTok{+}\NormalTok{(lambda}\SpecialCharTok{/}\NormalTok{mu)}\SpecialCharTok{\^{}}\DecValTok{1}\SpecialCharTok{/}\NormalTok{(}\DecValTok{1}\NormalTok{)}\SpecialCharTok{+}\NormalTok{((lambda}\SpecialCharTok{/}\NormalTok{mu)}\SpecialCharTok{\^{}}\DecValTok{2}\SpecialCharTok{/}\NormalTok{(}\DecValTok{2}\NormalTok{))}\SpecialCharTok{*}\NormalTok{(}\DecValTok{1}\SpecialCharTok{/}\DecValTok{1}\SpecialCharTok{{-}}\NormalTok{(lambda}\SpecialCharTok{/}\NormalTok{(s}\SpecialCharTok{*}\NormalTok{mu))))}
\NormalTok{lq}\OtherTok{=}\NormalTok{((p0}\SpecialCharTok{*}\NormalTok{((lambda}\SpecialCharTok{/}\NormalTok{mu)}\SpecialCharTok{\^{}}\DecValTok{2}\NormalTok{)}\SpecialCharTok{*}\NormalTok{(lambda}\SpecialCharTok{/}\NormalTok{mu)))}\SpecialCharTok{/}\NormalTok{(}\DecValTok{2}\SpecialCharTok{*}\NormalTok{(}\DecValTok{1{-}2}\NormalTok{)}\SpecialCharTok{\^{}}\DecValTok{2}\NormalTok{)}
\NormalTok{wq}\OtherTok{=}\NormalTok{lq}\SpecialCharTok{/}\DecValTok{20}
\FunctionTok{print}\NormalTok{(wq)}\CommentTok{\#El limite de clientes que se perdera es 0.06\%, con las condiciones actuales}
\end{Highlighting}
\end{Shaded}

\begin{verbatim}
## [1] 0.06666667
\end{verbatim}

\hypertarget{ejercicio-10}{%
\subsection{Ejercicio 10}\label{ejercicio-10}}

A los conductores nuevos se les pide pasar un examen por escrito, antes
de hacer las pruebas de manejo. Los exámenes escritos suelen hacerse en
el departamento de policía de la ciudad. Los registros de la ciudad de
Springdale indican que la cantidad promedio de exámenes escritos es de
100 por día de 8 horas. El tiempo necesario para contestar el examen es
de 30 minutos, más o menos. Sin embargo, la llegada real de los
aspirantes y el tiempo que tarda cada uno en contestar son totalmente
aleatorios.

Examenes escritos por hora \(100/8\) Tiempo de promedio por examenes
\(30\)

\begin{enumerate}
\def\labelenumi{\alph{enumi})}
\tightlist
\item
  La cantidad promedio de asientos que debe tener el departamento de
  policía en el salón de exámenes. \(p=\lambda/\mu\)=\((100/8)/2\)
\end{enumerate}

\begin{Shaded}
\begin{Highlighting}[]
\NormalTok{p}\OtherTok{=}\NormalTok{(}\DecValTok{100}\SpecialCharTok{/}\DecValTok{8}\NormalTok{)}\SpecialCharTok{/}\DecValTok{2}
\FunctionTok{print}\NormalTok{(p)}
\end{Highlighting}
\end{Shaded}

\begin{verbatim}
## [1] 6.25
\end{verbatim}

\textbf{la cantidad promedio de asientos es de 6.25 asi que debe de
haber 7 asientos}

\begin{enumerate}
\def\labelenumi{\alph{enumi})}
\setcounter{enumi}{1}
\tightlist
\item
  La probabilidad de que los aspirantes rebasen la cantidad promedio de
  asientos que hay en el salón de exámenes.
\end{enumerate}

\(p_n>=8=1-(\sum6.25^1/1!*e^-6.23+\sum6.25^2/2!*e^-6.23+...+\sum6.25^7/7!*e^-6.23)\)

\begin{Shaded}
\begin{Highlighting}[]
\NormalTok{e}\OtherTok{=}\FloatTok{2.718281828}
\NormalTok{pn}\OtherTok{=} \DecValTok{1}\SpecialCharTok{{-}}\NormalTok{((}\FloatTok{6.25}\SpecialCharTok{\^{}}\DecValTok{0}\SpecialCharTok{/}\DecValTok{1}\SpecialCharTok{*}\NormalTok{e}\SpecialCharTok{\^{}{-}}\FloatTok{6.25}\NormalTok{)}\SpecialCharTok{+}\FloatTok{6.25}\SpecialCharTok{\^{}}\DecValTok{1}\SpecialCharTok{/}\DecValTok{1}\SpecialCharTok{*}\NormalTok{e}\SpecialCharTok{\^{}{-}}\FloatTok{6.25+6.25}\SpecialCharTok{\^{}}\DecValTok{2}\SpecialCharTok{/}\DecValTok{2}\SpecialCharTok{*}\NormalTok{e}\SpecialCharTok{\^{}{-}}\FloatTok{6.23+6.25}\SpecialCharTok{\^{}}\DecValTok{3}\SpecialCharTok{/}\DecValTok{6}\SpecialCharTok{*}\NormalTok{e}\SpecialCharTok{\^{}{-}}\FloatTok{6.23+6.25}\SpecialCharTok{\^{}}\DecValTok{4}\SpecialCharTok{/}\DecValTok{24}\SpecialCharTok{*}\NormalTok{e}\SpecialCharTok{\^{}{-}}\FloatTok{6.23+6.25}\SpecialCharTok{\^{}}\DecValTok{5}\SpecialCharTok{/}\DecValTok{120}\SpecialCharTok{*}\NormalTok{e}\SpecialCharTok{\^{}{-}}\FloatTok{6.23+6.25}\SpecialCharTok{\^{}}\DecValTok{6}\SpecialCharTok{/}\DecValTok{720}\SpecialCharTok{*}\NormalTok{e}\SpecialCharTok{\^{}{-}}\FloatTok{6.23+6.25}\SpecialCharTok{\^{}}\DecValTok{7}\SpecialCharTok{/}\DecValTok{5040}\SpecialCharTok{*}\NormalTok{e}\SpecialCharTok{\^{}{-}}\FloatTok{6.23}\NormalTok{)}
\FunctionTok{print}\NormalTok{(pn)}
\end{Highlighting}
\end{Shaded}

\begin{verbatim}
## [1] 0.2770582
\end{verbatim}

La probabilidad de que los estudiantes rebasen el numero de asientos en
el salon es de 0.27705 c) La probabilidad de que en un día no se haga
examen alguno.

\begin{Shaded}
\begin{Highlighting}[]
\NormalTok{p0}\OtherTok{=}\FloatTok{6.25}\SpecialCharTok{\^{}}\DecValTok{0}\SpecialCharTok{/}\DecValTok{1}\SpecialCharTok{*}\NormalTok{e}\SpecialCharTok{\^{}{-}}\FloatTok{6.25}
\FunctionTok{print}\NormalTok{(p0)}
\end{Highlighting}
\end{Shaded}

\begin{verbatim}
## [1] 0.001930454
\end{verbatim}

la probabilidad de que no se presente ningun examen en un dia es de
0.001930454

\hypertarget{ejercicio-12}{%
\subsection{Ejercicio 12}\label{ejercicio-12}}

Metalco está contratando a un mecánico para un taller con 10 máquinas.
Se están examinando a dos candidatos. El primero puede reparar 5
máquinas por hora y gana 15 por hora. El segundo candidato, más hábil,
recibe 20 por hora y puede reparar 8 máquinas por hora. Metalco estima
que por cada máquina descompuesta se pierden \$50 por hora por falta de
producción. Suponiendo que las máquinas se descomponen siguiendo una
distribución de Poisson con una media de 3 por hora, y que el tiempo de
reparación tiene distribución exponencial, ¿a cuál persona se debe
contratar?

Sistema: taller, Clientes: máquinas que se descomponen, Servidores;2
mecánicos Tasa de llegadas (\(\lambda\))= 5 máquinas por hora, el
segundo 8 máquinas por hora Numero de servidores= 2 Tamaño de la cola=10

Calcule lo siguiente:

\begin{Shaded}
\begin{Highlighting}[]
\NormalTok{mu}\OtherTok{=}\DecValTok{5}\SpecialCharTok{/}\DecValTok{60}
\NormalTok{lambda}\OtherTok{=}\DecValTok{3}\SpecialCharTok{/}\DecValTok{60}
\NormalTok{s}\OtherTok{=}\DecValTok{2}
\NormalTok{mc}\OtherTok{=}\DecValTok{15}\SpecialCharTok{/}\DecValTok{60} \CommentTok{\#15 pesos por hora}
\NormalTok{mw}\OtherTok{=}\DecValTok{50}\SpecialCharTok{/}\DecValTok{60}  \CommentTok{\#Costo de máquina descompuesta}
\NormalTok{L}\OtherTok{=}\NormalTok{lambda}\SpecialCharTok{/}\NormalTok{(mu}\SpecialCharTok{{-}}\NormalTok{lambda)}
\NormalTok{L}
\end{Highlighting}
\end{Shaded}

\begin{verbatim}
## [1] 1.5
\end{verbatim}

\begin{Shaded}
\begin{Highlighting}[]
\CommentTok{\#Son maquinas del sistema}
\end{Highlighting}
\end{Shaded}

\begin{Shaded}
\begin{Highlighting}[]
\NormalTok{CTa}\OtherTok{=}\NormalTok{mc}\SpecialCharTok{*}\DecValTok{1}\SpecialCharTok{+}\NormalTok{mw}\SpecialCharTok{*}\FloatTok{1.5}
\FunctionTok{print}\NormalTok{(CTa)}
\end{Highlighting}
\end{Shaded}

\begin{verbatim}
## [1] 1.5
\end{verbatim}

\begin{Shaded}
\begin{Highlighting}[]
\CommentTok{\#Para el mecánico 2}
\NormalTok{mub}\OtherTok{=}\DecValTok{8}\SpecialCharTok{/}\DecValTok{60}
\NormalTok{lambda}\OtherTok{=}\DecValTok{3}\SpecialCharTok{/}\DecValTok{60}
\NormalTok{s}\OtherTok{=}\DecValTok{2}
\NormalTok{mcb}\OtherTok{=}\DecValTok{20}\SpecialCharTok{/}\DecValTok{60} \CommentTok{\#recibe 20 pesos por hora}
\NormalTok{mwb}\OtherTok{=}\DecValTok{50}\SpecialCharTok{/}\DecValTok{60}  \CommentTok{\#Costo máquina descompuesta}
\NormalTok{L}\OtherTok{=}\NormalTok{lambda}\SpecialCharTok{/}\NormalTok{(mub}\SpecialCharTok{{-}}\NormalTok{lambda)}
\CommentTok{\#Máquinas en el sistema}
\FunctionTok{print}\NormalTok{(L)}
\end{Highlighting}
\end{Shaded}

\begin{verbatim}
## [1] 0.6
\end{verbatim}

\begin{Shaded}
\begin{Highlighting}[]
\NormalTok{CTa}\OtherTok{=}\NormalTok{mcb}\SpecialCharTok{*}\DecValTok{1}\SpecialCharTok{+}\NormalTok{mwb}\SpecialCharTok{*}\FloatTok{0.6}
\FunctionTok{print}\NormalTok{(CTa)}
\end{Highlighting}
\end{Shaded}

\begin{verbatim}
## [1] 0.8333333
\end{verbatim}

\textbf{En conclusión, la mejor opción es contratar al segundo mecánico
pues su costo esperado por hora es menor en comparación del mecánico 1}*

\end{document}
